\documentclass{article}
\usepackage{amsmath,amssymb,amsthm,ascmac}
\usepackage{mathrsfs}
\usepackage{bm}
\usepackage[dvipdfmx]{graphicx,color}
\usepackage[utf8]{inputenc}
\usepackage[title=default]{phfnote}
\newcommand{\diff}{\mathrm{d}}
\newtheorem{define}{Definition}[section]
\newtheorem{theorem}{Theorem}[section]
\newtheorem{lem}{Lemma}[section]
\newtheorem{eg}{Example}[section]
\newtheorem{ans}{答え}
\newtheorem*{ans*}{答え}
\usepackage{pifont}
\newcommand{\cmark}{\ding{51}}%
\newcommand{\xmark}{\ding{55}}%
\newcommand{\red}[1]{\textcolor{red}{#1}}
\newcommand{\blue}[1]{\textcolor{blue}{#1}}
\newcommand{\esssup}{\mathop{\rm ess~sup}\limits}
\newcommand{\limltwo}{\mathop{\rm l.i.m.}\limits}
\newcommand{\km}{\mathrm{km}}
\newcommand{\supp}{\mathrm{supp}~}
\newcommand{\jpyen}{\mathrm{(\mbox{円})}}
\newcommand{\g}{\mathrm{(g)}}
\newcommand{\tori}{\mathrm{(\mbox{通り})}}
\newcommand*{\Perm}[2]{{}_{#1}\!P_{#2}}%
\newcommand{\vertiii}[1]{{\left\vert\kern-0.25ex\left\vert\kern-0.25ex\left\vert #1 
    \right\vert\kern-0.25ex\right\vert\kern-0.25ex\right\vert}}
\newcommand{\argmin}{\mathop{\rm arg~min}\limits}

% 式番号をsectionごとに取る
\makeatletter
\@addtoreset{equation}{section}
\def\theequation{\thesection.\arabic{equation}}% renewcommand でもOK
\makeatother

\title{GP and RKHS}
\author{米田亮介}
\date{\today}
\begin{document}
\maketitle

\section{準備}
\begin{define}[GP]
$\{X(t)\mid 0\leq t\leq T\}$が確率過程のとき、
$0\leq t\leq T$から任意に有限個$t_1,t_2,\dots t_n$を選んだときに、
$(X(t_1),X(t_2),\dots,X(t_n))^\mathsf{T}$が$n$次元の正規分布に従うならば
$\{X(t)\}$を\textbf{ガウス過程}(Gaussian process, GP)という。
\end{define}

$f$がGPのときに、各$x,x'$に対して
\begin{align}
    m(x)=\mathbb{E}[f(x)],\quad k(x,x')=\mathbb{E}[(f(x)-m(x))(f(x')-m(x'))]
\end{align}
で平均$m$と共分散$k$を定めて、
\begin{align}
    f\sim\mathcal{GP}(m,k)
\end{align}
と書くことにする。

GPを用いた回帰を復習する。
観測データ$\mathcal{D}=\{(x_{1},y_{1}),\dots,(x_{N},y_{N})\}$について
$x$と$y$の間に$y=f(x)$の関係があり、
GP$f\sim\mathcal{GP}(m,k)$から生成されているとする。
このとき事後分布は
\begin{align}
    f\mid\mathcal{D}\sim\mathcal{GP}(\bar{m},\bar{k})
\end{align}
でガウス過程になる。ここで、$k_{XX} = \{k(x_{i},x_{j})\}_{i,j},k_{xX}=(k(x,x_{1}),\dots,k(x,x_{N})),k_{Xx'}=k_{x'X}^{\mathsf{T}}$とおいて
\begin{align}
    &\bar{m}(x)=m(x)+k_{xX}k_{XX}^{-1}(\bm{y}-m_{X}),\\
    &\bar{k}(x,x')=k(x,x')-k_{xX}k_{XX}^{-1}k_{Xx'}
\end{align}
である。

\begin{define}[RKHS]
    集合$\mathcal{X}$上の\textbf{再生核ヒルベルト空間}(reproducing kernel Hilbert spcae, RKHS)とは、
    $\mathcal{X}$上の関数からなるヒルベルト空間$\mathcal{H}$で、
    任意の$x\in\mathcal{X}$に対して$k_{x}\in\mathcal{H}$があって
    \begin{align}
        \langle f,k_{x} \rangle_{\mathcal{H}} = f(x)
    \end{align}
    が任意の$f\in\mathcal{H}$に対して成り立つことである。
    $k(y,x)=k_{x}(y)$により定まるカーネル$k$を$\mathcal{H}$の\textbf{再生核}と呼ぶ。
\end{define}

再生核が一意であることとMoore--Aronszajnの定理によりカーネルとRKHSが1対1に対応することがわかる。

RKHSを導入することでGP回帰は非常に見通しがよくなる。

\begin{lem}
観測データ$\mathcal{D}=\{(x_{1},y_{1}),\dots,(x_{N},y_{N})\}$について
$x$と$y$の間に$y=f(x)$の関係があり、
$f$が平均$0$の$f\sim\mathcal{GP}(0,k)$から生成されているとする。
このとき、カーネル$k$に対応するRKHSを$\mathcal{H}_{k}$とおくと
\begin{align}
    \overline{m} = \argmin_{f\in\mathcal{H}_{k}}\|f\|_{\mathcal{H}_{k}},
    \textrm{ subject to } f(x_{i}) = y_{i},\quad i=1,\dots,N,
\end{align}
で表される。
\end{lem}

これは平均がRKHS$\mathcal{H}_{k}$に入ることを意味している。
そのためデータにのる関数の性質を決める上でカーネルを選び方は非常に重要である。

\section{RKHSの性質}
shift invariantなカーネル、すなわち、$k(x,y)=\Psi(x-y)$なる関数$\Psi$がある状況においては、
RKHSが定まる。

\begin{theorem}
    $k$を$\mathbb{R}^{d}$上のカーネルで、$k(x,y)=\Psi(x-y),\Psi\in C(\mathbb{R}^{d})\cap L^{1}(\mathbb{R}^{d})$であるとする。
    このとき、カーネル$k$のRKHSは
    \begin{align}
        \mathcal{H}_{k}=\{f\in L^{2}(\mathbb{R}^{d}) \cap C(\mathbb{R}^{d}) \mid \|f\|_{\mathcal{H}_{k}}<\infty \}
    \end{align}
    であり、$f,g\in \mathcal{H}_{k}$に対して内積は
    \begin{align}
        \langle f,g \rangle_{\mathcal{H}_{k}} = \frac{1}{(2\pi)^{d/2}}\int_{\mathbb{R}^{d}}\frac{\mathcal{F}[f](\omega)\overline{\mathcal{F}[g](\omega)}}{\mathcal{F}[\Psi](\omega)}\diff\omega
    \end{align}
    である。
\end{theorem}

ここで関数$f$に対するFourier変換を
\begin{align}
    \mathcal{F}[f](\omega)=\frac{1}{(2\pi)^{d/2}}\int_{\mathbb{R}^{d}}f(x)e^{-ix^{\mathsf{T}}\omega}\diff x
\end{align}
で定めた。

具体例を見ていこう。

\begin{eg}[RBFカーネル]
RBFカーネルは
\begin{align}
    k_{\gamma}(x,y)=\Psi(x-y)=\exp(-\|x-y\|^{2}/\gamma^{2})
\end{align}
で与えられる。
ここでガウス関数のFourier変換はガウス関数であったことを思い出そう。すなわち、
\begin{align}
    \mathcal{F}[\Psi](\omega)=\frac{\gamma^{d}}{2^{d/2}}\exp(-\gamma^{2}\|\omega\|^{2}/4)
\end{align}
である。
よってRBFカーネルのRKHSは
\begin{align}
    & \mathcal{H}_{k_{\gamma}}=\{f\in L^{2}(\mathbb{R}^{d})\cap C(\mathbb{R}^{d}) \mid \|f\|_{\mathcal{H}_{k_{\gamma}}} < \infty \},\\
    & \|f\|^{2}_{\mathcal{H}_{k_{\gamma}}}=\frac{1}{\gamma^{d}\pi^{d/2}}\int_{\mathbb{R}^{d}}|\mathcal{F}[f](\omega)|^{2}\exp(\gamma^{2}\|\omega\|^{2}/4)\diff\omega
\end{align}
である。
特に$\|f\|_{\mathcal{H}_{k_{\gamma}}}$が収束するとき、$\mathcal{F}[f](\omega)$は$\omega$は
指数関数的に減衰することがわかる。
よってPaley–Wienerの定理から$f\in\mathcal{H}_{k_{\gamma}}$は解析的であることがわかる。
これがRBFカーネルから生成されるGPが$C^{\infty}$級である、と言われることの直感的な説明である。
(より正確にはGPのサンプルパスは確率0で$\mathcal{H}_{k_{\gamma}}$に属することが示せてしまうので
もう少し高度な数学を用いて議論する必要がある。)
\end{eg}

\begin{eg}[Laplaceカーネル]
Laplaceカーネルは
\begin{align}
    k_{\alpha}(x,y)=\Psi(x-y)=\exp(-\alpha|x-y|)
\end{align}
で与えられる。Fourier変換を行えば
\begin{align}
    \mathcal{F}[\Psi](\omega)=\sqrt{\frac{2}{\pi}}\frac{\alpha}{\omega^{2}+\alpha^{2}}
\end{align}
であるから、RKHSは
\begin{align}
    & \mathcal{H}_{k_{\alpha}}=\{f\in L^{2}(\mathbb{R}^{d})\cap C(\mathbb{R}^{d}) \mid \|f\|_{\mathcal{H}_{k_{\alpha}}} < \infty \},\\
    & \|f\|^{2}_{\mathcal{H}_{k_{\gamma}}}=\frac{1}{\pi}\int_{\mathbb{R}^{d}}|\mathcal{F}[f](\omega)|^{2}(\omega^{2}+\alpha^{2})\diff\omega
\end{align}
である。
微分のFourier変換について$\mathcal{F}[f'](\omega)=-i\omega\mathcal{F}[f](\omega)$が成り立つこと思い出すと、
\begin{align}
    \|f\|_{\mathcal{H}_{k_{\alpha}}}<\infty
    \Leftrightarrow \|f\|_{L^{2}}<\infty, \|Df\|_{L^{2}}<\infty
\end{align}
となるから$\mathcal{H}_{k_{\alpha}}$はSobolev空間$H^{1}(\mathbb{R})$である。
\end{eg}

\begin{eg}[Mat\'ernカーネル]
詳しくは書かないが、Mat\'ernカーネル$k_{\alpha,h}$に対応するRKHSは
Sobolev空間$W_{2}^{\alpha+d/2}(\mathcal{X})$に同型であることがわかる。
\end{eg}

\section{周期カーネル}
周期カーネルは
\begin{align}
    k_{\theta}(x,y)=\Psi(x-y)=\theta_{0}\exp(\theta_{1}\cos(x-y))
\end{align}
で与えられる。これまでの違いは考える定義域が$\mathbb{R}$から$\mathbb{S}^{1}$に移ったことである。
これに伴ってRKHS内のFourier変換はFourier級数として理解する必要がある。(ここの正確な証明があるのかは知らないです。)

周期関数$f$のFourier級数展開の第$n$係数を
\begin{align}
    \hat{f}_{n}=\frac{1}{\sqrt{2\pi}}\int_{\mathbb{S}^{1}}f(x)e^{-inx}\diff x
\end{align}
で定める。すると周期カーネルのRKHSは
\begin{align}
    \mathcal{H}_{k_{\theta}}=\{f\in L^{2}(\mathbb{S}^{1}) \cap C(\mathbb{S}^{1}) \mid \|f\|_{\mathcal{H}_{k_{\theta}}}<\infty \}
\end{align}
であり、内積は
\begin{align}
    \langle f,g \rangle_{\mathcal{H}_{k_{\theta}}}
    =\frac{1}{\sqrt{2\pi}}\sum_{n\in\mathbb{Z}}\frac{\hat{f}_{n}\overline{\hat{g}_{n}}}{\hat{\Psi}_{n}}
\end{align}
となるだろう。
ノルムに着目してみよう。
\begin{align}
    \|f\|_{\mathcal{H}_{k_{\theta}}}=\frac{1}{\sqrt{2\pi}}\sum_{n\in\mathbb{Z}}\frac{|\hat{f}_{n}|^{2}}{\hat{\Psi}_{n}}<\infty
\end{align}
である。特に$\Psi$は解析的な関数であるからPaley–Wienerの定理によりある$C,\varepsilon>0$が存在して十分大きな$n$で
\begin{align}
|\hat{\Psi}_{n}|\leq Ce^{-\varepsilon|n|}
\end{align}
のようにFourier係数が指数関数的な減衰を見せる。
これより$|\hat{f}_{n}|$も$n$によって指数関数的な減衰を見せることがわかる。
またPaley–Wienerの定理を使えば$f\in\mathcal{H}_{k_{\theta}}$が解析的であることがわかる。
(これは全くもって厳密な証明ではなく、直感的な説明を書いているだけです。)

\begin{thebibliography}{99}
    \bibitem{gp} M. Kanagawa, P. Hennig, D. Sejdinovic, and B. K. Sriperumbudur,
    Gaussian Processes and Kernel Methods: A Review on Connections and Equivalences,
    arXiv: 1807.02582.
    \bibitem{kernel} カーネル法入門, 福水健次, 朝倉書店, 2010.
  \end{thebibliography}

\end{document}