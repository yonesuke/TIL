\documentclass{jsarticle}
\usepackage{amsmath,amssymb}
\usepackage{ascmac}
\usepackage[dvipdfmx]{graphicx,color}
\usepackage{theorem}
\newtheorem{theorem}{定理}
\newtheorem{proof}{証明}
\def\qed{\hfill $\Box$}

\begin{document}
\title{Ott--Antonsen縮約\cite{ott2008low,ott2009long}による蔵本モデルの解析}
\author{米田亮介}
\maketitle	

\begin{abstract}
\textbf{Ott--Antonsen縮約}\cite{ott2008low,ott2009long}
は無限次元に拡張された蔵本モデルを有限次元に縮約する手法の一つであり、
蔵本モデルの自己無矛盾方程式、中心多様体縮約に並んで用いられる解析手法である。
ここではOtt--Antonsen縮約を用いた具体的な計算過程とその結果を述べたい。
\end{abstract}

\section{蔵本モデル}



\bibliography{oa} %hoge.bibから拡張子を外した名前
\bibliographystyle{junsrt} %参考文献出力スタイル

\end{document}