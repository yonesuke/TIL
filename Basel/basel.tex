\documentclass[fleqn]{jsarticle}
\usepackage{euler}
\usepackage{amsmath,amssymb}
\usepackage{ascmac}
\usepackage[dvipdfmx]{graphicx,color,hyperref}
\usepackage{pxjahyper}
\usepackage{url}
\usepackage{framed}
\definecolor{shadecolor}{gray}{0.80}

\title{Basel問題}
\author{米田亮介\footnote{yonedaryosuke@amp.i.kyoto-u.ac.jp}}

\begin{document}
\maketitle

\abstract{
Basel問題は、平方数の逆数の和はいくらになるのか?という問題である。
1644年にPietro Mengoliにより提起され、1735年にLeonhard Eulerによって解かれた。
それによると、値は
\[\sum_{n=1}^{\infty}\frac{1}{n^{2}}=\frac{\pi^{2}}{6}\]
となる。
Basel問題の証明に関してはEuler以降様々に提唱されている。
この記事では調べられるだけすべてのBasel問題の証明を集めていきたいと思う。
}

\tableofcontents

\newpage

\section{Fourier級数展開を用いる方法}
\begin{shaded}
$f(t)$を$2\pi$周期の周期関数とする。
$f$の\textbf{(複素)フーリエ級数展開}は、
\begin{align}
&f(t)\sim\sum_{n=-\infty}^{\infty}f_{n}e^{int},\\
&f_{n}=\frac{1}{2\pi}\int_{-\pi}^{\pi}f(t)e^{-int}dt
\end{align}
で定められる。
\end{shaded}
区間$[-\pi,\pi]$で
\begin{align}
	f(t)=t^{2}
\end{align}
となる$2\pi$周期の周期関数$f$を考える。$f$を複素フーリ級数展開することで次の恒等式を得る。
\begin{align}
	t^{2}=\frac{\pi^{2}}{3}+\sum_{n\ne 0,n=-\infty}^{\infty}\frac{2(-1)^{n}}{n^{2}}e^{-int}
	\label{eq:t^2-fourier}
\end{align}
ここに$t=\pi$を代入すると、
\begin{align}
	\pi^{2}&=\frac{\pi^{2}}{3}+\sum_{n\ne 0,n=-\infty}^{\infty}\frac{2(-1)^{n}}{n^{2}}e^{-in\pi}\\
	&=\frac{\pi^{2}}{3}+\sum_{n\ne 0,n=-\infty}^{\infty}\frac{2(-1)^{n}}{n^{2}}(-1)^{n}\\
	&=\frac{\pi^{2}}{3}+\sum_{n\ne 0,n=-\infty}^{\infty}\frac{2}{n^{2}}\\
	&=\frac{\pi^{2}}{3}+4\sum_{n=1}^{\infty}\frac{1}{n^{2}}
\end{align}
となる。これより、
\begin{align}
	\sum_{n=1}^{\infty}\frac{1}{n^{2}}=\frac{\pi^{2}}{6}
\end{align}
であることが示された\footnote{\url{http://yonesuke1729.hatenablog.com/entry/2016/06/17/130215}}。

\newpage

\section{Fourier級数展開を用いる方法}
式~\eqref{eq:t^2-fourier}において$t=0$を代入すると、
\begin{align}
	0=\frac{\pi^{2}}{3}+4\sum_{n=1}^{\infty}\frac{(-1)^{n}}{n^{2}}
\end{align}
が得られる。この式を整理すると、
\begin{align}
	\sum_{n=1}^{\infty}\frac{(-1)^{n+1}}{n^{2}}=\frac{\pi^{2}}{12}
\end{align}
となる。

一方で、$\sum\frac{1}{n^2}$を求めたいが、これは$n\geq 2$において、
\begin{align}
	\frac{1}{n^{2}}<\frac{1}{n(n-1)}=\frac{1}{n-1}-\frac{1}{n}
\end{align}
となることを用いると、$\sum\frac{1}{n^2}<2$が示される。
よって、この級数は正項級数で上に有界であるから収束する。
この収束値を$\alpha$とおく。このとき、

\begin{align}
	\alpha-\frac{\pi^{2}}{12}
	&=\sum_{n=1}^{\infty}\frac{1}{n^2}-\sum_{n=1}^{\infty}\frac{(-1)^{n+1}}{n^{2}}\\
	&=\sum_{n=1}^{\infty}\left\{\frac{1}{n^{2}}-\frac{(-1)^{n+1}}{n^{2}}\right\}\\
	&=\sum_{n:偶数}\frac{2}{n^{2}}\\
	&=\sum_{m=1}^{\infty}\frac{2}{(2m)^{2}}\\
	&=\frac{1}{2}\sum_{n=1}^{\infty}\frac{1}{n^{2}}=\frac{\alpha}{2}
\end{align}
これより、
\begin{align}
	\alpha=\sum_{n=1}^{\infty}\frac{1}{n^{2}}=\frac{\pi^{2}}{6}
\end{align}
が示された\footnote{\url{http://yonesuke1729.hatenablog.com/entry/2016/06/17/130215}}。

\newpage

\section{中間値の定理を用いる方法}

この証明のアイデアは、$n\geq 0$で成り立つ次の恒等式
\footnote{恒等式~\eqref{eq:identity}は$x=2m\pi,m\in\mathbb{Z}$で値を持たないが、
その場合は極限値を考える。}
\begin{align}
	\frac{1}{2}+\sum_{k=1}^{n}\cos(kx)=\frac{\sin\left(n+\frac{1}{2}\right)x}{2\sin\frac{x}{2}}
	\label{eq:identity}
\end{align}
と中間値の定理を用いるものである。具体的には次の命題を用いる。
\begin{shaded}
	$f:[a,b]\to\mathbb{R}$を$[a,b]$上で連続とし、
	$g:[a,b]\to\mathbb{R}$を$[a,b]$上で非負関数で積分可能($g\in\mathrm{L}^{1}[a,b]$)とする。
	このとき、ある$\xi\in[a,b]$が存在して、次が成立する。
	\begin{align}
		\int_{a}^{b}f(x)g(x)dx=f(\xi)\int_{a}^{b}g(x)dx
	\end{align}
\end{shaded}

このとき、式~\eqref{eq:identity}の両辺に$x^{2}-2x$を掛けて$[0,\pi]$上で積分すると、
左辺は、
\begin{align}
	&\int_{0}^{\pi}\left\{\frac{1}{2}+\sum_{k=1}^{n}\cos(kx)\right\}(x^{2}-2\pi x)dx\\
	=&\frac{1}{2}\int_{0}^{\pi}(x^{2}-2\pi x)dx+\sum_{k=1}^{n}\int_{0}^{\pi}\cos(kx)(x^2-2\pi x)dx\\
	=&\frac{1}{2}\left[\frac{1}{6}x^{3}-\frac{\pi}{2}x^{2}\right]_{0}^{\pi}
	+\sum_{k=1}^{n}\int_{0}^{\pi}\left(\frac{1}{k}\sin(kx)\right)'(x^{2}-2\pi x)dx\\
	=&-\frac{\pi^{3}}{3}
	+\sum_{k=1}^{n}\left\{\left[\frac{1}{k}\sin(kx)(x^2-2\pi x)\right]_{0}^{\pi}
	-\frac{2}{k}\int_{0}^{\pi}\sin(kx)(x-\pi)dx\right\}\\
	=&-\frac{\pi^{3}}{3}
	-\sum_{k=1}^{n}\frac{2}{k}\int_{0}^{\pi}\left(-\frac{1}{k}\cos(kx)\right)'(x-\pi)dx\\
	=&-\frac{\pi^{3}}{3}+\sum_{k=1}^{n}\frac{2}{k^{2}}
	\left\{\left[\cos(kx)(x-\pi)\right]_{0}^{\pi}
	-\int_{0}^{\pi}\cos(kx)dx
	\right\}\\
	=&-\frac{\pi^{3}}{3}+\sum_{k=1}^{n}\frac{2\pi}{k^{2}}
\end{align}
となる。右辺については、
\begin{align}
	&\int_{0}^{\pi}(x-2\pi)\frac{\frac{x}{2}}{\sin\frac{x}{2}}\sin\left(\left(n+\frac{1}{2}\right)x\right)dx\\
	=&\int_{0}^{\pi}(x-2\pi)\frac{\frac{x}{2}}{\sin\frac{x}{2}}\left\{-\frac{1}{n+\frac{1}{2}}\cos\left(\left(n+\frac{1}{2}\right)x\right)\right\}'dx\\
	=&-\left[(x-2\pi)\frac{\frac{x}{2}}{\sin\frac{x}{2}}\frac{1}{n+\frac{1}{2}}\cos\left(\left(n+\frac{1}{2}\right)x\right)\right]_{0}^{\pi}
	+\frac{1}{n+\frac{1}{2}}\int_{0}^{\pi}\left\{(x-2\pi)\frac{\frac{x}{2}}{\sin\frac{x}{2}}\right\}'\cos\left(\left(n+\frac{1}{2}\right)x\right)dx\\
	=&\frac{-2\pi}{n+\frac{1}{2}}
	+\frac{\cos\left(\left(n+\frac{1}{2}\right)\xi_{n}\right)}{n+\frac{1}{2}}
	\int_{0}^{\pi}\left\{(x-2\pi)\frac{\frac{x}{2}}{\sin\frac{x}{2}}\right\}'dx\\
	=&\frac{-2\pi}{n+\frac{1}{2}}
	+\frac{\cos\left(\left(n+\frac{1}{2}\right)\xi_{n}\right)}{n+\frac{1}{2}}\left(2\pi-\frac{\pi^{2}}{2}\right)\to0
\end{align}
となることがわかる。
よって、両辺の極限$n\to\infty$を取ると、
\begin{align}
	-\frac{\pi^{3}}{3}+\sum_{k=1}^{\infty}\frac{2\pi}{k^{2}}=0
\end{align}
となり、これを整理すると、
\begin{align}
	\sum_{k=1}^{\infty}\frac{1}{k^{2}}=\frac{\pi^{2}}{6}
\end{align}
となり、Basel問題が示された\footnote{Samuel G. Moreno,
	A One-Sentence and Truly Elementary Proof of the Basel Problem,
	arXiv:\href{https://arxiv.org/abs/1502.07667}{1502.07667}.}。


\end{document}