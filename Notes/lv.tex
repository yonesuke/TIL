\documentclass{jsarticle}
\usepackage[dvipdfmx]{graphicx,color}
\usepackage{ulem}
\usepackage{ascmac}
\usepackage{pgf,tikz}
\usepackage{here}
%\usepackage{subcaption}
\usepackage{amsmath,amssymb}
\usepackage{ascmac}
\usepackage{comment}
\usepackage{theorem}
\newtheorem{theorem}{定理}
\newtheorem{proof}{証明}
\newtheorem{remark}{Remark}
\usepackage{framed}
\def\qed{\hfill $\Box$}
\definecolor{shadecolor}{gray}{0.80}

%\usepackage{showkeys}

\begin{document}
\title{数理解析特論レポート}
\author{米田亮介}
\maketitle
\section{Darboux変換}
\begin{shaded}
\begin{enumerate}
\item 差分演算子$L,B$をそれぞれ
$L=e^{\partial/\partial n}+u_{n}e^{-\partial/\partial n},
B=-u_{n}u_{n-1}e^{-2\partial/\partial n}$で定める。
このとき、関数$\varphi_{n}(t)$に対する線形方程式
\begin{eqnarray}
L\varphi_{n}(t)=(\lambda+1/\lambda)\varphi_{n}(t),\quad
\frac{d}{dt}\varphi_{n}(t)=B\varphi_{n}(t)
\end{eqnarray}
の両立条件から、無限格子上の連続時間発展方程式であるLotka--Voltera(LV)方程式
\begin{eqnarray}
\frac{d}{dt}u_{n}=u_{n}(u_{n+1}-u_{n-1})
\end{eqnarray}
が導かれることを示せ。
ここで、$\lambda+1/\lambda$は$L$の固有値を与える定数。
\item LV方程式の自明な解$u=1$をseed(種)として、
Darboux変換を$k$回適用することで得られる差分作用素を
$L^{(k)}=e^{\partial/\partial n}+u_{n}^{(k)}e^{-\partial/\partial n}$
で表す。
このとき、$u_{n}^{(1)}$および$u_{n}^{(2)}$を求めよ。
\end{enumerate}
\end{shaded}
\end{document}