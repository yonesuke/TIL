\documentclass{jsarticle}
\usepackage{euler}
\usepackage{amsmath,amssymb}
\usepackage{ascmac}
\usepackage[dvipdfmx]{graphicx,color,hyperref}
\usepackage{pxjahyper}
\usepackage{url}
\usepackage{framed}
\definecolor{shadecolor}{gray}{0.80}
\begin{document}
\setcounter{section}{2}

\setcounter{subsection}{2}
\subsection{}
\begin{shaded}
$(\partial+x)^{-1}$を計算せよ。
\end{shaded}
\begin{align}
Y=\sum_{n=1}^{\infty}h_{n}\partial^{-n}
\end{align}
として$(\partial+x)Y$を計算すると
\begin{align*}
(\partial+x)\sum_{n=0}^{\infty}h_n\partial^{-n}
&=\sum_{n=1}^{\infty}\left(\frac{\partial h_{n}}{\partial x}+xh_{n}\right)\partial^{-n}+h_n\partial^{-n+1}\\
&=h_{1}+\sum_{n=1}^{\infty}\left(h_{n+1}+\frac{\partial h_{n}}{\partial x}+xh_{n}\right)\partial^{-n}
\end{align*}
である。
$(\partial+x)Y=1$とすると、$h_{n}$に関する次の漸化式を得る。
\begin{align}
\begin{aligned}
&h_{1}=1,\\
&h_{n+1}=-\frac{\partial h_{n}}{\partial x}-xh_{n}.
\end{aligned}
\end{align}
\setcounter{subsection}{5}
\subsection{}
\begin{shaded}
$\dfrac{\partial u}{\partial x_{5}}$は何になるか。
\end{shaded}
(2.12)から
\begin{align}
\frac{\partial u}{\partial x_{5}}=K_{5}(u)=-[P,(P^{5/2})_{+}]
\end{align}
である。ここで、(2.5)をさらに低次まで計算すると、
\begin{align*}
(\partial^{2}+u)^{1/2}=\partial+\frac{1}{2}u\partial^{-1}-\frac{1}{4}u_{x}\partial^{-2}+\left(\frac{u_{xx}}{8}-\frac{u^{2}}{8}\right)\partial^{-3}+\left(\frac{3}{8}uu_{x}-\frac{1}{16}u_{xxx}\right)\partial^{-4}+\cdots
\end{align*}
である。これより、
\begin{align*}
\left(P^{5/2}\right)_{+}&=\left((\partial^{2}+u)^{2}\left[\partial+\frac{1}{2}u\partial^{-1}-\frac{1}{4}u_{x}\partial^{-2}+\left(\frac{u_{xx}}{8}-\frac{u^{2}}{8}\right)\partial^{-3}+\left(\frac{3}{8}uu_{x}-\frac{1}{16}u_{xxx}\right)\partial^{-4}+\cdots\right]\right)_{+}\\
&=\partial^{5}+\frac{5}{2}u\partial^{3}+\frac{15}{4}u_{x}\partial^{2}+\left(\frac{25}{8}u_{xx}+\frac{15}{8}u^{2}\right)\partial+\left(\frac{15}{16}u_{xxx}+\frac{15}{8}uu_{x}\right)
\end{align*}
であるから、交換子積は
\begin{align*}
&\left[\partial^{2}+u,\partial^{5}+\frac{5}{2}u\partial^{3}+\frac{15}{4}u_{x}\partial^{2}+\left(\frac{25}{8}u_{xx}+\frac{15}{8}u^{2}\right)\partial+\left(\frac{15}{16}u_{xxx}+\frac{15}{8}uu_{x}\right)\right]\\
=&\partial^{2}\left(\frac{15}{16}u_{xxx}+\frac{15}{8}uu_{x}\right)
-\left[\partial^{5}+\frac{5}{2}u\partial^{3}+\frac{15}{4}u_{x}\partial^{2}+\left(\frac{25}{8}u_{xx}+\frac{15}{8}u^{2}\right)\partial\right]u\\
=&-\frac{1}{16}(u_{5x}+10u_{3x}u+20u_{2x}u_{x}+30u_{x}u^{2})
\end{align*}
となる(交換子積の結果が関数になることを利用すると計算量が減らせる)。以上より、
\begin{align}
\frac{\partial u}{\partial x_{5}}=\frac{1}{16}(u_{5x}+10u_{3x}u+20u_{2x}u_{x}+30u_{x}u^{2})
\end{align}
である。

\subsection{}
\begin{shaded}
(2.16)の$f_{1},f_{2}$と(2.18)の$w_{1},w_{2}$の間の関係を導け。
\end{shaded}
$L\circ M=M\circ\partial$を用いる。ここで、
\begin{align}
\begin{aligned}
&L=\partial+f_{1}\partial^{-1}+f_{2}\partial^{-2}+\cdots,\\
&M=1+w_{1}\partial^{-1}+w_{2}\partial^{-2}+w_{3}\partial^{-3}+\cdots
\end{aligned}
\end{align}
である。負の階数の微分は消えることにすると、
\begin{align*}
L\circ M&=\partial+w_{1}+\left(\frac{\partial w_{1}}{\partial x}+w_{2}+f_{1}\right)\partial^{-1}+\left(\frac{\partial w_{2}}{\partial x}+w_{3}+f_{1}w_{1}+f_{2}\right)\partial^{-2}+\cdots,\\
M\circ\partial&=\partial+w_{1}+w_{2}\partial^{-1}+w_{3}\partial^{-2}+\cdots
\end{align*}
となる。係数比較をすると、
\begin{align}
\begin{aligned}
&\frac{\partial w_{1}}{\partial x}+f_{1}=0,\\
&\frac{\partial w_{2}}{\partial x}+f_{1}w_{1}+f_{2}=0
\end{aligned}
\end{align}
が得られる。

\subsection{}
\begin{shaded}
$M^{-1}=1+v_{1}\partial^{-1}+v_{2}\partial^{-2}+\cdots$を求めよ。
\end{shaded}
$M\circ M^{-1}=1$より、
\begin{align*}
1=1+(w_{1}+v_{1})\partial^{-1}+\sum_{k=2}^{\infty}\left(w_{k}+v_{k}+\sum_{i=1}^{k-1}w_{i}v_{k-i}\right)\partial^{-k}
\end{align*}
なので係数比較すると、
\begin{align}
\begin{aligned}
&v_{1}=-w_{1},\\
&v_{k}=-w_{k}-\sum_{i=1}^{k-i}w_{i}v_{k-i}\ (k\ge2)
\end{aligned}
\end{align}
となる。

\subsection{}
\begin{shaded}
(2.25),(2.26)は
\begin{align*}
&\frac{\partial \log\tau}{\partial x_{1}}=-w_{1},\\
&\frac{\partial \log\tau}{\partial x_{2}}=-2w_{2}+w_{1}^{2}-\frac{\partial w_{1}}{\partial x_{1}}
\end{align*}
と書き直せる。この2つの関係式が両立していることを示せ。
\end{shaded}
\subsection{}
\begin{shaded}
(2.28)を導け。
\end{shaded}
\end{document}