\documentclass[10pt, pre, twocolumn, showpacs, aps]{revtex4-1}
\usepackage{amsmath,amssymb}
\usepackage[dvipdfmx]{graphicx,color}
\usepackage{ulem}
\usepackage{pgf,tikz}
\usepackage{here}
%\usepackage{subcaption}
\usepackage{comment}
\usepackage{theorem}
\newtheorem{theorem}{定理}
\newtheorem{proof}{証明}
\newtheorem{remark}{Remark}
\newcommand{\blue}[1]{\textcolor{blue}{#1}}
\def\qed{\hfill $\Box$}

\begin{document}
\title{Note on "Isotropic--Nematic Phase Transitions in Gravitational Systems"}
\author{Ryosuke Yoneda}
\maketitle

\section{Introduction}
現在見つかっている銀河系の多くは中心に超大質量ブラックホール(SMBH)
があることが観測によりわかっている。
そして、SMBHの回りを運動する星たちにが存在して、銀河系が形成されるのである。
そんな訳で、SMBH回りの星たちの運動がどのように進行していくのかを調べることは
銀河系が形成されることを調べる上でとても重要である。
実際にいくつかの研究があり、SMBH回りの星たちの運動について、
タイムスケールに関する次のことが分かっている。
\begin{itemize}
\item はじめに、一番小さなタイムスケールにおいては星がSMBH回りの周期運動を行う。
\item 次に短いタイムスケールとして、近日点移動がある。
\item そして、次のタイムスケールでは、軌道角運動量の大きさ、
楕円軌道の形、すなわち、楕円の長軸と離心率が保存された状態で
軌道平面の傾きが緩和する状態が訪れる。
このプロセスをvector resonant relaxation(VRR)と呼び、
本研究の主題にもなっている。
\item 先程は軌道角運動量の大きさや楕円軌道の形は固定されていたが、	
このタイムスケールではそれも緩和する方向に進む。
このプロセスをscalar resonant relaxation(SRR)と呼ぶ。
\item 最後に訪れる現象として二体緩和がある。
このときには、エネルギーもが緩和する方向に進むのだが、
実はこのタイムスケールは宇宙の年齢をも超えることが計算により分かっている。
このような観点から二体緩和は考えることが少ないのかもしれない。	
\end{itemize}
表にまとめるとこんな感じ。
\begin{table}[H]
\begin{tabular}{|c|l|}\hline
	$t_{\rm orb}\sim1$-$10^4$ years & 軌道の一周期分の時間\\ \hline
	$t_{\rm aps}\sim10^3$-$10^5$ years & 近日点移動の時間\\ \hline
	\blue{$t_{\rm vrr}\sim10^6$-$10^7$ years} & \blue{軌道平面の変化が緩和する時間}\\ \hline
	$t_{\rm srr}\sim10^8$-$10^{10}$ years & 角運動量の大きさが緩和する時間\\ \hline
	$t_{\rm 2-body}\sim10^9$-$10^{10}$ years & エネルギーが緩和する時間\\ \hline
\end{tabular}
\end{table}
実際に私達の銀河系を調べてみると、
SMBH回りの星たちの軌道平面の傾きについて、
75%くらいの星たちはSMBH回りを薄いdiskのような形で同じ軌道平面を描いている。
一方、残りの25%の星たちはいろいろバラバラな方向を向いていることがわかった。
このような軌道平面の傾きに関するばらつきはVRRのタイムスケールで現れると思える、
そして、実際にモデルを立てて検証したいというのが物理学者の性である。
次のセクション以降で実際にハミルトニアンを設定して、
統計力学的な処方箋に基づいて平衡状態を求め、
頑張って数値計算もした様子を紹介したい。

\begin{remark}
ここで、統計力学を用いて、という話をしたが、
重力系において、統計力学を用いるのは本来は筋が通らない話をしていることには注意したほうが良い。
なぜなら、重力系には真の平衡状態などというものは存在しないからである。
これは非常に簡単な説明をしようとすれば、
ハミルトニアンを設定すると重力のポテンシャルの項として
$1/r$の項が現れる。
カノニカル分布における平衡状態は"雑に言うと"
exponentialの肩にハミルトニアンを載せることで得られるので、
重力の項の影響で$e^{1/r}$という非常に扱いづらい計算が生じるのである。
このような観点から重力系で統計力学を扱うのは難しいのである。
なので、このような話を聞いたときにはまずは疑ってかかるのが吉である。
今回のようになんらかの近似を行って統計力学的な熱平衡状態を得ているのだ、
と考えるのが良いのかもしれない。	
\end{remark}

\section{ハミルトニアンの設定}
SMBH回りの星の集団的運動を調べたい。
今回は特にタイムスケールで言うとVRRについて調べることにする。
もう一度まとめておくと、VRRにおいては、
SMBH回りの星たちは各々軌道平面上を運動しているとしていて、
さらにその星の角運動量と楕円軌道の形、すなわち長軸と離心率は保存している、
とする。
このとき、VRRのタイムスケールで動くのは、
各々の楕円軌道の傾き、すなわち角運動量の向きである。
イントロでも述べたように、ここでは統計力学を考えたいので、
何はともあれ、まずはハミルトニアンを立てましょう。

SMBHの質量を$M_{\bullet}$とする。
この質量はSMBHの回りを回る星たちの質量の総和に比べてもかなり重い、
という仮定を置こう。
そして、SMBHは$\mathbb{R}^{3}$上の原点に固定されているとしよう。
あと、これは少々強い仮定になるかもしれないが、
SMBH回りの星たちの質量とその楕円軌道の形、角運動量の大きさは
すべて同じである、としよう。
本来は異なるものを考えるべきであり、論文中でもそうしているのだが、
論文では異なる場合についてはfuture workとして、
質量等が全て同じであるものに専念して考えることにしている。
まあ、異なるケースを考えるための足掛りになるし、
toy modelと思って見たときには別にそんな不思議なことをしている訳でもない。
ということで、今回は星たちの質量とその楕円軌道の形、角運動量の大きさは
すべて同じである、と仮定するのである。
このとき、質量を$m$、楕円軌道の形を特徴づける長軸を$a$、
離心率を$e$、そして角運動量の大きさを$\ell$としよう。
星が$N$個あるとすると、もうハミルトニアンを定義することができる。
\begin{align}
H=\sum_{i=1}^{N}\frac{\pmb{p}_{i}^{2}}{2m}
-\sum_{i=1}^{N}\frac{GM_{\bullet}m}{\|\pmb{q}_{i}\|}
-\frac{1}{2}\sum_{i\ne j}\frac{Gm^{2}}{\|\pmb{q}_{i}-\pmb{q}_{j}\|}.
\end{align}
こんな風になるのである。
やったー、これでもう機械的に計算ができるぜ!!というわけではない。
イントロでも述べたように、ハミルトニアンの$1/r$の項
が厄介ものなのである。
どのように対処していくのかを今から見ていこう。

まずはじめに、各々の星たちのKeplarian energyが保存することが分かる。
もっと具体的に言うと、星$i$に対して、
\begin{align}
\frac{\pmb{p}_{i}^{2}}{2m}-\frac{GM_{\bullet}m}{\|\pmb{q}_{i}\|}=-\frac{GM_{\bullet}m}{2a}
\end{align}
と計算できるのである。
いま、楕円軌道の長軸はVRRにおいては固定している、と考えるのであった。
そうすると、このKeplarian energyが保存することが分かる。
なので、まずこの項を落とすことができる。
\begin{remark}
「この項が保存するから、ハミルトニアンから落としてしまおう」
と非常に簡単に言ってしまっているが、本当はちゃんと考えないと行けない。
いや、ちゃんと考えなくても良いのだが、ハミルトニアンとはなんぞや、
ということはしっかり考えなくては行けない。
ハミルトニアンには様々な役割があって、その最たる例は運動方程式を導出するためのもの、である。
その場合、保存量だからその項を落としてしまおう、なんて無茶なことをしてはいけない。
そんなことをしてしまったら時間依存しないハミルトニアンの場合、
それ自身が保存量になるからハミルトニアンを落としてしまおう、
なんて言ってしまうと、運動方程式自体を立てられなくなってしまうからである。
しかし、この場合はどうなんだろうか、ハミルトニアンの値が大事なんだろうか、
そこらへんがあまり分かっていない。
そうだとするならば、最初のKeplarian energyの項は完全なる定数になるので、
落としてもいい、というのは理解できる。
ここはあまりきちんとした理解が出来ていない。
今後研究していくにつれて、その"気持ち"が分かることを期待して、
今は論文に従って、読み進めて行こう。
\end{remark}

こうすると、考えるべきハミルトニアンは次の形になる。
\begin{align}
H=-\frac{1}{2}\sum_{i\ne j}\frac{Gm^{2}}{\|\pmb{q}_{i}-\pmb{q}_{j}\|}
=-\sum_{i<j}\frac{Gm^{2}}{\|\pmb{q}_{i}-\pmb{q}_{j}\|}.
\end{align}

次に、VRRが起こるタイムスケールについて考えてみよう。
このタイムスケールは軌道の周期の時間と、近日点移動の時間比べると、
圧倒的に長いのであった。
なのでVRRのプロセスを考える場合については軌道平均と近日点移動の平均
をとってしまおう。
この近似はBorn--Openheimer近似に通づるものがあるが、
今回はこのことはまたの機会に置いておこう。
ともかく、2つの平均を取ることにしよう。
そしてさらに、球面調和関数による展開、を行おう。
そうしてガリガリと計算を進めて行くと、
ハミルトニアン$H_{VRR}$は次のように書けることが分かる。
\begin{align}
H_{VRR}=-\sum_{i<j}\sum_{l=1}^{\infty}\mathcal{J}_{ijl}P_{l}(\pmb{n}_{i}\cdot\pmb{n}_{j})
\end{align}
ここで$P_{l}$はルジャンドル多項式で、球面調和関数が退化していって現れたものである。
また、$\pmb{n}_{i}$は星$i$の軌道平面の向きの単位成分を表している。
具体的に計算しようとすると、関係式として次のようになる。
\begin{align}
\pmb{q}_{i}\times\pmb{p}_{i}=\ell\pmb{n}_{i}.
\end{align}
あと、係数である$\mathcal{J}_{ijl}$は次のようになる。
\begin{align}
\mathcal{J}_{ijl}=\frac{Gm^{2}}{a}P_{l}(0)^{2}s_{ijl}\alpha_{ij}^{l}.
\end{align}
ここに現れる係数の細かな説明についてはしない。
(というか今の段階でまだそんなに計算を追うことができていない。)
ここでとにかく大事なのは$l$の扱いである。
$l$が奇数のときは$P_{l}(0)$が消えるので考えなくて良い。
偶数のときは$l\geq 2$を考えるのだが、
数値計算によると、$l=2$の影響がドミナント、圧倒的になるらしいのだ。
この結果に関してはとりあえずのところ信じることにしよう。
このとき、$l=2$に対応する状態を四重極子という。
なのでこの論文では$l=2$の部分だけを取り出すことに着目して
quadrapole approximationなんて言い方をしている。
そういった意味では、この論文はひとつには銀河系の複雑な構造を説明したい、
というモチベーションがあるのだが、裏を返せば、
このquadrapole approximationの正当性について考える論文とも言えそうだ。
実際に数値シミュレーションを行うと、ある程度、変なものが出てきているので、
そういった意味ではこの近似は良かったのかな、と思えるのだ。

$l=2$だけを取り出したとき、ハミルトニアンをもう少しexplicitに書き下してみよう。
\begin{align}
H=-\frac{3J}{2}\sum_{i,j=1}^{N}\left[(\pmb{n}_{i}\cdot\pmb{n}_{j})^{2}-\frac{1}{3}\right]
\end{align}
\end{document}