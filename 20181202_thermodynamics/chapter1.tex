\documentclass{jsarticle}
\usepackage{amsmath,amssymb}
\usepackage[dvipdfmx]{graphicx,color}
\usepackage{ulem}
\usepackage{pgf,tikz}
\usepackage{here}
%\usepackage{subcaption}
\usepackage{comment}
\usepackage{theorem}
\newtheorem{theorem}{定理}
\newtheorem{proof}{証明}
\newtheorem{remark}{Remark}
\newcommand{\blue}[1]{\textcolor{blue}{#1}}
\def\qed{\hfill $\Box$}

\begin{document}
\title{第1章\\熱力学の紹介と下準備}
\date{2018/12/02}
\maketitle
\section{ミクロ・マクロと陥りやすい幻想}
電子、原子、分子のような基本構成要素に関する系を\textbf{ミクロな系}という。
ミクロな系での運動は量子論で記述される。
一方で、原子とかが莫大な数が集まって相互作用しているような系を\textbf{マクロな系}という。
熱力学が考えるのは、マクロな系のマクロな振る舞いである。

物理学者が考えることとして、
すべての自然現象はミクロな世界の基本法則(量子論)で支配されるだろう、というものがある。
しかし、実際的に考えると、マクロな系を量子論を用いて完全に理解することは不可能である。
様々に要因は絡んでいるが、考える方程式の次元が高くなりすぎるのと非線形性が
効きすぎていることがその一つと考えられる。
ここらへんは微分方程式の可積分性とかともつながる話である。
例えば、重力系について考えてみよう。
二体問題の場合は、簡単に解ける。
軌道が具体的に楕円になることとかはよく知られた結果である。
しかし、三体問題になった瞬間に物事は非常に複雑になる。
これはポアンカレ以来の難問である。
そして多くの場合には非可積分であることが知られている。

そんな訳で、量子論を使ってマクロな系の微分方程式を立てて実際に解こうなんて
ことが如何に無謀であるかが分かると思う。
なので、マクロな系を考えるにはマクロな系の理論を打ち立てる必要があるのだ。
\section{熱力学の意義}
熱力学の歴史的な経緯については山本さんの本を読むと良いらしい。
山本さんは高校のときに読んでいた駿台の『新物理入門』を書いた先生でもある。
磁力と重力について書いた本もあるので読んでみたいと思う。

前のセクションで書いたようにマクロな系の理論を打ち立てたい。
このとき、一番望ましいのは複雑な運動の中から本質だけを抽出した、
マクロ系専用の理論体系を構築することである。
系すべての運動を記述するのは不可能だが、本質的な部分だけは記述出来てしまう
ような理論が熱力学なのである。
これはちょうど蔵本モデルを考えるときに、個々の振動子の位相がどうなっているのか、
なんて興味はなくて、そのかわりに振動子の重心を表す秩序変数についての運動だけを
取り出す、ということにも似ている。
また、蔵本モデルの無限次元系である連続の式について、
自明解の周りで中心多様体縮約を行うのも似た話で、
これも十分時間が経ったときの運動しか興味がなくて、
そのときの運動は中心多様体に沿って運動しているはずだ、
と思うことで無限次元の偏微分方程式を有限次元の常微分方程式に落とし込むことができるのだ。
このように、本質だけを抜き出してその理論を構築する、というのは非常に大事な考え方である。

あと、熱力学という学問によってわかったことには、
\textbf{操作限界}を数式で簡潔に記述出来たことにある。
これは熱力学第2法則で述べるところである。

最後に物理の基礎理論の大雑把な分類を表~\ref{table:classification}にまとめておく。
\begin{table}[H]
\begin{center}
\caption{物理学の基礎理論の大雑把な分類}
\label{table:classification}
\begin{tabular}{|c|c|}\hline
ミクロ系の理論 & 力学、真空中の電磁気学、電磁気学、場の量子論、$\cdots$ \\
\hline
ミクロとマクロをつなぐ理論 & 平衡系の統計力学、非平衡系の統計力学(未) \\
\hline
\blue{マクロ系の平衡状態とその間の遷移の理論} & \blue{平衡系の熱力学} \\
\hline
マクロ系の非平衡状態の理論 & 流体力学、物質中の電磁気学、非平衡系の熱力学(未)\\ 
\hline
\end{tabular}
\end{center}
\end{table}
\end{document}