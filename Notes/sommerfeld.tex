\documentclass{jsarticle}
\usepackage{amsmath,amssymb}
\usepackage[T1]{fontenc}
\usepackage{newpxtext, newpxmath}
\usepackage{ascmac}
\usepackage{fancybox}
\usepackage[dvipdfmx]{graphicx,color}
\usepackage{theorem}
\newtheorem{theorem}{定理}
\newtheorem{proof}{証明}
\newtheorem{eg}{例}
\def\qed{\hfill $\Box$}
\def\d{\textrm{d}}

\begin{document}
\title{Sommerfeld expansion}
%\author{米田亮介}
\maketitle	

\section{Introduction}
1粒子状態の粒子数$n_{\tau}$の取りうる値は量子力学によると次の2つの場合しかない:
\begin{itemize}
\item $n_{\tau}=0,1$ in \textbf{Fermi--Dirac ensembles},
\item $n_{\tau}=0,1,2,\cdots,\infty$ in \textbf{Bose--Einstein ensembles}.
\end{itemize}
Fermi統計に従う粒子を\textbf{Fermi粒子}、
Bose統計に従う粒子を\textbf{Bose粒子}という。

Fermi粒子が独立に運動している場合、一つのエネルギー準位$\varepsilon$にある粒子の数は
\begin{align*}
f(\varepsilon)=\frac{1}{e^{\beta(\varepsilon-\mu)}+1}
\end{align*}
で与えられる。ここで、$\mu$は化学ポテンシャルである。
Fermi分布に従う物理量$g(\varepsilon)$の統計平均は
\begin{align*}
\int_{-\infty}^{\infty}g(\varepsilon)f(\varepsilon)\d\varepsilon
\end{align*}
であり、特に低温付近では、
\begin{center}
\shadowbox{$\displaystyle
\int_{-\infty}^{\infty}g(\varepsilon)f(\varepsilon)\d\varepsilon
=\int_{-\infty}^{\mu}g(\varepsilon)\d\varepsilon
+\frac{\pi^{2}g^{(1)(\mu)}}{6}(kT)^{2}
+\frac{7\pi^{4}g^{(3)}(\mu)}{360}(kT)^{4}+\cdots
$}
\end{center}
となることが知られている。この展開を\textbf{低温展開}(\textbf{Sommerfeld expansion})
といい、いろいろな応用が知られている。

\section{Proof}
はじめに、次の公式を証明する。
\begin{center}
\shadowbox{$\displaystyle-\int_{-\infty}^{\infty}\varphi(\varepsilon)\frac{\d f}{\d\varepsilon}\d\varepsilon
=\varphi(\mu)+\frac{\pi^{2}\varphi^{(2)}(\mu)}{6}(kT)^{2}
+\frac{7\pi^{4}\varphi^{(4)}(\mu)}{360}(kT)^{4}+\cdots.$}
\end{center}
ただし、$\varphi(-\infty)=0$とする。

$\varphi(\varepsilon)$を$\varepsilon=\mu$まわりで展開して、
\begin{align*}
\varphi(\varepsilon)=\varphi(\mu)+\sum_{k=1}^{\infty}\frac{\varphi^{(k)}(\mu)}{k!}(\varepsilon-\mu)^{k}
\end{align*}
代入すると、
\begin{align*}
-\int_{-\infty}^{\infty}\varphi(\varepsilon)\frac{\d f}{\d\varepsilon}\d\varepsilon
=-\varphi(\mu)\int_{-\infty}^{\infty}\frac{\d f}{\d\varepsilon}\d\varepsilon
-\sum_{k=1}^{\infty}\frac{\varphi^{(k)}(\mu)}{k!}\int_{-\infty}^{\infty}
(\varepsilon-\mu)^{k}\frac{\d f}{\d\varepsilon}\d\varepsilon
\end{align*}
となる。ここで初項について、
\begin{align*}
\int_{-\infty}^{\infty}\frac{\d f}{\d\varepsilon}\d\varepsilon
=\left[f\right]_{-\infty}^{\infty}=-1
\end{align*}
である。また、第2項の$k$が奇数の場合については$\frac{\d f}{\d\varepsilon}$が
$\mu$周りで偶関数であることから、
\begin{align*}
\int_{-\infty}^{\infty}(\varepsilon-\mu)^{k}\frac{\d f}{\d\varepsilon}\d\varepsilon=0
\end{align*}
である。最後に、$k$が偶数のときは、
\begin{align*}
\int_{-\infty}^{\infty}(\varepsilon-\mu)^{k}\frac{\d f}{\d\varepsilon}\d\varepsilon
&=-\frac{1}{k_{B}T}\int_{-\infty}^{\infty}\frac{(\varepsilon-\mu)^{k}e^{\beta(\varepsilon-\mu)}}{(e^{\beta(\varepsilon-\mu)}+1)^{2}}\d\varepsilon\\
&=-(k_{B}T)^{k}\int_{-\infty}^{\infty}\frac{x^{k}e^{x}}{(e^{x}+1)^{2}}\d x\\
&=-2k!(1-2^{-k+1})\zeta(k)(k_{B}T)^{k}
\end{align*}
である(\ref{sec:integrals}を見よ)。ここで、$\zeta(z)$は\textbf{Riemann zeta関数}
\begin{align*}
\zeta(z)=\sum_{n=1}^{\infty}\frac{1}{n^{z}}
\end{align*}
である。これより、
\begin{align*}
-\int_{-\infty}^{\infty}\varphi(\varepsilon)\frac{\d f}{\d\varepsilon}\d\varepsilon
=\varphi(\mu)+\sum_{r=1}^{\infty}2(1-2^{1-2r})\zeta(2r)\varphi^{(2r)}(\mu)(k_{B}T)^{2r}
\end{align*}
となる。特に、$r=1,2$を考えると、
\begin{align*}
-\int_{-\infty}^{\infty}\varphi(\varepsilon)\frac{\d f}{\d\varepsilon}\d\varepsilon
=\varphi(\mu)+\frac{\pi^{2}\varphi^{(2)}(\mu)}{6}(kT)^{2}
+\frac{7\pi^{4}\varphi^{(4)}(\mu)}{360}(kT)^{4}+\cdots
\end{align*}
となることが示された。

次に、
\begin{align*}
g(\varepsilon)=\varphi'(\varepsilon)
\end{align*}
を代入すると、
\begin{align*}
&-\int_{-\infty}^{\infty}\varphi(\varepsilon)\frac{\d f}{\d\varepsilon}\d\varepsilon=-\left[\varphi(\varepsilon)f(\varepsilon)\right]_{-\infty}^{\infty}
+\int_{-\infty}^{\infty}\varphi'(\varepsilon)f(\varepsilon)\d\varepsilon
=\int_{-\infty}^{\infty}g(\varepsilon)f(\varepsilon)\d\varepsilon,\\
&\varphi(\mu)=\int_{-\infty}^{\mu}\varphi'(\varepsilon)\d\varepsilon
=\int_{-\infty}^{\mu}g(\varepsilon)\d\varepsilon
\end{align*}
であるから、
\begin{align*}
\int_{-\infty}^{\infty}g(\varepsilon)f(\varepsilon)\d\varepsilon
=\int_{-\infty}^{\mu}g(\varepsilon)\d\varepsilon
+\frac{\pi^{2}g^{(1)}(\mu)}{6}(kT)^{2}
+\frac{7\pi^{4}g^{(3)}(\mu)}{360}(kT)^{4}+\cdots
\end{align*}
となり、低温展開を示すことが出来た。

\appendix
\section{Integral formulas}
\label{sec:integrals}
低温展開の証明の中で出てきた積分公式をいくらか示しておく。
\begin{itemize}
\item 
\shadowbox{
	$\displaystyle
\int_{0}^{\infty}\frac{x^{n}e^{x}}{(e^{x}+1)^{2}}\d x=n!(1-2^{1-n})\zeta(n)
	$
}
\begin{align*}
&\int_{0}^{\infty}\frac{x^{n}e^{x}}{(e^{x}+1)^{2}}\d x\\
=&\int_{0}^{\infty}x^{n}\frac{\d}{\d x}\left(\frac{-1}{e^{x}+1}\right)\d x
=n\int_{0}^{\infty}\frac{x^{n-1}}{e^{x}+1}\d x\\
=&n\int_{0}^{\infty}\frac{x^{n-1}e^{-x}}{1+e^{-x}}\d x
=n\int_{0}^{\infty}x^{n-1}e^{-x}\left[\sum_{k=0}^{\infty}(-1)^{k}e^{-kx}\right]\d x\\
=&n\sum_{k=0}^{\infty}(-1)^{k}\int_{0}^{\infty}x^{n-1}e^{-(k+1)x}\d x
=n\sum_{k=0}^{\infty}(-1)^{k}\int_{0}^{\infty}\frac{t^{n-1}}{(k+1)^{n-1}}e^{-t}\frac{\d t}{k+1}\\
=&n\sum_{k=0}^{\infty}\frac{(-1)^{k}}{(k+1)^{n}}\int_{0}^{\infty}t^{n-1}e^{-t}\d t
=n\Gamma(n)\sum_{k=0}^{\infty}\frac{(-1)^{k}}{(k+1)^{n}}\\
=&n!\sum_{k=0}^{\infty}\frac{(-1)^{k}}{(k+1)^{n}}
=n!\left(\sum_{l:{\rm odd}}\frac{1}{l^{n}}-\sum_{l:{\rm even}}\frac{1}{l^{n}}\right)\\
=&n!\left(\sum_{l=1}^{\infty}\frac{1}{l^{n}}-2\sum_{l:{\rm even}}\frac{1}{l^{n}}\right)
=n!(1-2^{1-n})\sum_{l=1}^{\infty}\frac{1}{l^{n}}
=n!(1-2^{1-n})\zeta(n)
\end{align*}
\item
\shadowbox{
	$\displaystyle
\zeta(2n)=(-1)^{n+1}\frac{B_{2n}(2\pi)^{2n}}{2(2n)!}
	$
}

有名事実(?)なので気が向いたら証明するかも。
\end{itemize}

\begin{thebibliography}{99}
  \item 久保亮五編『大学演習 熱学$\cdot$統計力学』修訂版(裳華房, 1998)
\end{thebibliography}

\end{document}