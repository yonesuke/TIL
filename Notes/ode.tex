\documentclass{jsarticle}
\usepackage[dvipdfmx]{graphicx,color}
\usepackage{ulem}
\usepackage{ascmac}
\usepackage{pgf,tikz}
\usepackage{here}
%\usepackage{subcaption}
\usepackage{amsmath,amssymb}
\usepackage{ascmac}
\usepackage{comment}
\usepackage{theorem}
\newtheorem{theorem}{定理}
\newtheorem{proof}{証明}
\newtheorem{remark}{Remark}
\usepackage{framed}
\def\qed{\hfill $\Box$}
\definecolor{shadecolor}{gray}{0.80}

%\usepackage{showkeys}

\begin{document}
\title{微分積分学続論II}
\author{米田亮介}
\maketitle
\section{11/14の微積続論II小テスト}
\begin{shaded}
次の微分方程式を解け。
\begin{enumerate}
\item 
\begin{align}
\frac{dx}{dt}=x\log t
\end{align}
\item 
\begin{align}
\frac{dx}{dt}=x\log t+t^{t}
\end{align}
\end{enumerate}
\end{shaded}
\begin{enumerate}
\item 
$x=0$は解である。
$x\ne 0$のとき両辺を変数分離すると、
\begin{align}
\frac{dx}{x}=\log tdt
\end{align}
であり、これを積分すると、
\begin{align}
\log |x|=t\log t-t+C'
\end{align}
である。ここで$C'$は積分定数である。
これより、解は
\begin{align}
|x(t)|=Ce^{t\log t-t}=Ce^{-t}t^{t}
\end{align}
となる。ここで$C>0$は定数。
$x=0$の解もまとめると、任意の$C\in\mathbb{R}$に対して
\begin{align}
x(t)=Ce^{-t}t^{t}
\end{align}
となる。
\item
この問題には定数変化法を使う。
つまり、解の形を
\begin{align}
x(t)=C(t)e^{-t}t^{t}
\end{align}
の形に決め打ちして、$C(t)$を求めるという算段である。
このとき、
\begin{align}
C'(t)e^{-t}t^{t}+x\log t=x\log t+t^{t}
\end{align}
である。これより、$C(t)$に関しての微分方程式
\begin{align}
C'(t)=e^{t}
\end{align}
が得られる。故に
\begin{align}
C(t)=e^{t}+C
\end{align}
となる。ここで$C$は積分定数である。
以上より、はじめの微分方程式の解は
\begin{align}
x(t)=(e^{t}+C)e^{-t}t^{t}=t^{t}+Ce^{-t}t^{t}
\end{align}
と求まった。
\end{enumerate}

\begin{comment}
\begin{remark}
微分方程式を解くときに思考停止で変数分離法とか定数変化法を使ってるけど、
解が本当にこれだけなのか?ということは常にしっかりと考えるべきことなのかもしれない。
初期値を与えたときに、解の存在と一意性を提示してくれるものは
ベクトル場のリプシッツ連続性であった。
なので、ひとたびベクトル場のリプシッツ連続性を確認することができたならば、
変数分離法や定数変化法は強力な手法になるのである。
今回の問題では自励系の微分方程式ではないので、
補助変数を用意して強引に自励系の多変数微分方程式系にする必要がある。
このとき、高次元にベクトル場が与えられたときにも解の存在と一意性を決めるのは
リプシッツ連続性になるのだろうか?
このことについても少し調べてみる必要がありそうだ。
\end{remark}
\end{comment}

\section{テキスト第3章問5}
\begin{shaded}
$a_{j}(t)(j=1,2)$と$q(t)$を$t$のある関数として2階非同次方程式
\begin{align}
x''+a_{1}(t)x'+a_{2}(t)x=q(t),\quad t\ne -1
\end{align}
を考える。$x=t,t^{2},t^{3}$が解であるとき、次の問いに答えよ。
\begin{enumerate}
\item 関数$a_{j}(t)$と$q(t)$を定めよ。
\item 一般解を求めよ。
\end{enumerate}
\end{shaded}
\begin{enumerate}
\item $x=t,t^{2},t^{3}$を代入して連立方程式を解くだけ。
\begin{align}
a_{1}(t)=-\frac{2(2t-1)}{t(t-1)},\\
a_{2}(t)=\frac{2(3t^{2}-3t+1)}{t^{2}(t-1)^{2}},\\
q(t)=\frac{2t}{(t-1)^{2}}
\end{align}
\item 非同時方程式の一般解は、同次方程式の一般解と特殊解の和になる。
そのために同次方程式
\begin{align}
x''+a_{1}(t)x'+a_{2}(t)x=0
\end{align}
を解きたいが、そのまま解くのは難しい。
そこで次のように考える。
非同次方程式の解$t,t^{2}$を同次方程式の基本解$x_{1},x_{2}$
を用いて表してみる。
特殊解は$t^{3}$であるから、
\begin{align}
&t=c_{1}x_{1}+c_{2}x_{2}+t^{3},\\
&t^{2}=c_{1}'x_{1}+c_{2}'x_{2}+t^{3}
\end{align}
とかける。
ここで$c_{1},c_{2},c_{1}',c_{2}'$は適当な定数である。
両辺を引くと、
\begin{align}
t-t^{2}=(c_{1}-c_{1}')x_{1}+(c_{2}-c_{2}')x_{2}
\end{align}
となる。
これより、$t-t^{2}$は基本解の線形和で表せることがわかった。
よって、$t-t^{2}$も同次方程式の解である。
同様の考えによって、$t-t^{3}$も同次方程式の解である。

$t-t^{2},t-t^{3}$はそれぞれ独立であるから、
同次方程式の基本解になる。
非同次方程式の特殊解として$t^{3}$を選ぶと、一般解は
\begin{align}
c_{1}(t-t^{2})+c_{2}(t-t^{3})+t^{3}
\end{align}
になる。
\end{enumerate}
\begin{remark}
最後の答えを書くところで非同次方程式の特殊解として、$t^{3}$を選んだが、
$t,t^{2}$でも良い。
どれを選んでも結局$c_{1},c_{2}$の任意性から解空間はおなじになる。
\end{remark}

\begin{comment}
\begin{remark}
よくある問題は特殊解を一つ与えた上で、
非同次方程式の一般解を求めさせる問題だが、
この問題は特殊解を(複数個)
先に与えている、という点において目新しい。
よくよく考えると、微分方程式を与えると、
(一般解に対応する)解空間を考えることができるが、
この問題は、解空間の中の有限個の点から解空間全体を復元できることも主張している。
線形空間であれば、次元の数のぶんだけ解を用意すればよいが、
今回は線形空間でもない、という点においても非常に面白い問題であると思う。
\end{remark}
\end{comment}

\begin{remark}
\sout{問題文のはじめにある$t\ne -1$が意味不明。}
$t\ne 0,1$の間違い。$t=0,1$では与えられた特解が独立でなくなるからだめ。
\end{remark}

\begin{remark}
同次方程式を直接求める方法もまとめておく。
同次方程式は、
\begin{align}
x''+a_{1}x'+a_{2}x=x''-\frac{2(2t-1)}{t(t-1)}x'+\frac{2(3t^{2}-3t+1)}{t^{2}(t-1)^{2}}=0
\end{align}
である。線形の2階常微分方程式に対するチルンハウス変換\footnote{3次方程式$ax^{3}+bx^{2}+cx+d=0$を$x^{3}+px+q=0$という形に変換することをチルンハウス変換という。今回はその類似でチルンハウス変換という言い方をした。}を施そう。
方程式を$x=e^{-\frac{1}{2}\int^{t} a_{1}(t)dt}\eta$で変数変換すると、
\begin{align}
\eta''=r\eta
\end{align}
に変換されることが知られている(確認せよ。)。ここで、
\begin{align}
r=\frac{1}{4}a_{1}^{2}+\frac{1}{2}a_{1}'-a_{2}
\end{align}
である。
今の場合、$r$を具体的に計算すると、$r=0$となる。
よって、変数変換された同次方程式は
\begin{align}
\eta''=0
\end{align}
を満たす。
これは簡単に解くことが出来て、任意定数$c_{1},c_{2}$を用いて、
$\eta=c_{1}t+c_{2}$となる。
今度、逆変換を行うと、
\begin{align}
x=c_{1}te^{-\frac{1}{2}\int^{t} a_{1}(t)dt}+c_{2}e^{-\frac{1}{2}\int^{t} a_{1}(t)dt}
\end{align}
である。ここで、
\begin{align}
e^{-\frac{1}{2}\int^{t} a_{1}(t)dt}=
\begin{cases}
t(t-1) & \text{$t>1$または$t<0$}\\
-t(t-1) & \text{$0<t<1$}
\end{cases}
\end{align}
となるが、符号は$c_{1},c_{2}$の任意性の方に吸収出来るので無視すると、
\begin{align}
x=c_{1}t^{2}(t-t)+c_{2}t(t-1)
\end{align}
である。
ここで、$(c_{1},c_{2})=(1,0),(0,1)$という風にとると、
$x=t^{2}(t-1),t(t-1)$はそれぞれ同次方程式の解であり、しかも独立である。
以上から、同次方程式の一般解は、
\begin{align}
x=c_{1}t^{2}(t-1)+c_{2}t(t-1)
\end{align}
となる。
\end{remark}


\section{テキスト第3章問6}
\begin{shaded}
2階同次方程式
\begin{align}
x''-\frac{1}{t+1}x'+\frac{1}{(t+1)^{2}}x=0,\quad t\ne -1
\label{eq:toi6}
\end{align}
に対して次の問いに答えよ。
\begin{enumerate}
\item $x=t+1$が解であることを示せ。
\item 定数変化法を用いて一般解を求めよ。
\item 初期条件$x(0)=x_{0},x'(0)=v_{0}$を満足する解を求めよ。
\end{enumerate}
\end{shaded}
\begin{enumerate}
\item 代入するだけ。
\item 与えられた微分方程式は$x$について線形だから、
$x=t+1$が解であれば$x=C(t+1)$も解である。
このとき、定数変化法を用いて一般解を求めたいので、
\begin{align}
x=C(t)(t+1)
\end{align}
として、$C(t)$を求める。
式~\eqref{eq:toi6}に代入すると、$C(t)$に関する微分方程式
\begin{align}
(t+1)C''+C'=0
\end{align}
が得られる。まず、$C'$についての微分方程式だと思って、変数分離法で解くと、
\begin{align}
C'(t)=\frac{c_{2}}{t+1}
\end{align}
が得られる。$c_{2}$は任意定数。よって、
\begin{align}
C(t)=c_{1}+c_{2}\log |t+1|
\end{align}
が得られる。$c_{1}$も任意定数\footnote{積分定数の決め方がずいぶんと恣意的になってしまった。。。}。
$t\ne-1$で$\log|t+1|$もきちんと定義されるので問題ない。
以上より、一般解は、
\begin{align}
x=c_{1}(t+1)+c_{2}(t+1)\log|t+1|
\end{align}
である。
\item $t=0$における$x,x'$の値を実際に計算して任意定数を求めればよい。
\begin{align}
x_{0}=x(0)=c_{1},\\
v_{0}=x'(0)=C(0)+C'(0)=c_{1}+c_{2}
\end{align}
であるから、$c_{1}=x_{0},c_{2}=v_{0}-x_{0}$と求まる。
以上より、初期条件$x(0)=x_{0},x'(0)=v_{0}$を満足する解は、
\begin{align}
x=x_{0}(t+1)+(v_{0}-x_{0})(t+1)\log|t+1|
\end{align}
である。
\end{enumerate}

\section{テキスト第3章問7}
\begin{shaded}
2階非同次方程式
\begin{align}
x''-\frac{2}{t+1}x'+\frac{2}{(t+1)^{2}}x=t+1,\quad t\ne -1
\label{eq:toi7}
\end{align}
に対して次の問いに答えよ。
\begin{enumerate}
\item 定数変化法を用いて一般解を求めよ(問2を参照せよ)。
\item 初期条件$x(0)=x_{0},x'(0)=v_{0}$を満足する解を求めよ。
\end{enumerate}
\end{shaded}
\begin{enumerate}
\item まず、同次方程式に対する解として$x=t+1,(t+1)^{2}$がある。
次に、非同次方程式の一般解を求めるためには、非同次方程式の特殊解を
ひとつ求めれば良い。そのために、$x=t+1$
に対して定数変化法を用いることを考える。
先程と同じように$x=C(t)(t+1)$として、式~\eqref{eq:toi7}に代入
すると、$C(t)$に関する微分方程式
\begin{align}
C''=1
\end{align}
が得られる。この微分方程式の解のひとつは、$C(t)=\frac{1}{2}t^{2}$
となるので、特殊解は
\begin{align}
x=\frac{1}{2}t^{2}(t+1)^{2}
\end{align}
である。以上より、式~\eqref{eq:toi7}の一般解は
\begin{align}
x=\frac{1}{2}t^{2}(t+1)+c_{1}(t+1)+c_{2}(t+1)^{2}
\end{align}
である。
\item $t=0$における$x,x'$の値を実際に計算して任意定数を求めればよい。
\begin{align}
x_{0}=x(0)=c_{1}+c_{2},\\
v_{0}=x'(0)=c_{1}+2c_{2}
\end{align}
なので、$c_{1}=2x_{0}-v_{0},c_{2}=v_{0}-x_{0}$が分かる。
よって、初期条件$x(0)=x_{0},x'(0)=v_{0}$を満足する解は
\begin{align}
x=\frac{1}{2}t^{2}(t+1)+(2x_{0}-v_{0})(t+1)+(v_{0}-x_{0})(t+1)^{2}
\end{align}
である。
\end{enumerate}

\section{テキスト第3章問12}
\begin{shaded}
次の微分方程式の一般解を求めよ。
\begin{enumerate}
\item $x''+x=1$
\item $x''-x'-2x=t^{2}$
\item $x''+x'-2x=(3t^{4}+4t^{3})e^{t}$
\item $x''+x'+x=te^{-t}$
\end{enumerate}
\end{shaded}
\begin{enumerate}
\item 対応する同次方程式の特性方程式は
\begin{align}
\lambda^{2}+1=0
\end{align}
である。これを解くと、$\lambda=\pm i$であり、基本解は
\begin{align}
x=\cos t,\sin t
\end{align}
である。また、特性解は$x=1$である。
よって、一般解は
\begin{align}
x=c_{1}\cos t+c_{2}\sin t+1
\end{align}
である。$c_{1},c_{2}$は任意定数である。
\item 対応する同次方程式の特性方程式の特性指数は
$\lambda=2,-1$なので、基本解は
\begin{align}
x=e^{2t},e^{-t}
\end{align}
である。特性解として$c_{1}t^{2}+c_{2}t+c_{3}$の形のものを考える。
微分方程式に特性解を代入すると、
\begin{align}
-2c_{1}t^{2}+(-2c_{1}+-2c_{2})t+(2c_{1}-c_{2}-2c_{3})
=t^{2}
\end{align}
となるので、係数比較をして連立方程式を解くと、
\begin{align}
c_{1}=-\frac{1}{2},c_{2}=\frac{1}{2},c_{3}=-\frac{3}{4}
\end{align}
である。以上より、一般解は
\begin{align}
x=c_{1}e^{2t}+c_{2}e^{-t}
-\frac{1}{2}t^{2}+\frac{1}{2}t-\frac{3}{4}
\end{align}
である。
\item 対応する同次方程式の特性方程式の特性指数は
$\lambda=-2,1$なので、基本解は
\begin{align}
x=e^{-2t},e^{t}
\end{align}
である。次に非同次方程式の解を
\begin{align}
x=u_{1}e^{-2t}+u_{2}e^{t}
\end{align}
と置いてみる。また、
\begin{align}
u_{1}'e^{-2t}+u_{2}'e^{t}=0
\end{align}
を仮定する。微分方程式にこれを代入すると、$u_{1}',u_{2}'$について、
\begin{align}
u_{1}'e^{-2t}+u_{2}'e^{t}=0,\\
-2u_{1}'e^{-2t}+u_{2}'e^{t}=(3t^{4}+4t^{3})e^{t}
\end{align}
となるから、
\begin{align}
u_{1}'=-\left(t^{4}+\frac{4}{3}t^{3}\right)e^{3t},\\
u_{2}'=t^{4}+\frac{4}{3}t^{3}
\end{align}
それぞれを積分すると、
\begin{align}
u_{1}=-\frac{1}{3}e^{3t}t^{4},\\
u_{2}=\frac{1}{5}t^{5}+\frac{1}{3}t^{4}
\end{align}
である。よって、特殊解は
\begin{align}
x=\frac{1}{5}t^{5}e^{t}
\end{align}
である。以上より一般解は、
\begin{align}
x=c_{1}e^{-2t}+\left(c_{2}+\frac{1}{5}t^{5}\right)e^{t}
\end{align}
である。
\item 対応する同次方程式の特性方程式の特性指数は
$\lambda=-\frac{1}{2}\pm\frac{\sqrt{3}}{2}i$なので、基本解は
\begin{align}
x=e^{-\frac{t}{2}}\cos\left(\frac{\sqrt{3}}{2}t\right),
e^{-\frac{t}{2}}\sin\left(\frac{\sqrt{3}}{2}t\right)
\end{align}
である。次に非同次方程式の解を
\begin{align}
x=(c_{1}t+c_{2})e^{-t}
\end{align}
と仮定する。非同次方程式に代入して係数比較すると、
\begin{align}
c_{1}=1,\\
-c_{1}+c_{2}=0
\end{align}
となる。これより$c_{1}=c_{2}=1$なので、特殊解は
\begin{align}
x=(t+1)e^{-t}
\end{align}
である。以上より一般解は
\begin{align}
x=e^{-\frac{t}{2}}\left[c_{1}\cos\left(\frac{\sqrt{3}}{2}t\right)
+c_{2}\sin\left(\frac{\sqrt{3}}{2}t\right)\right]
+(t+1)e^{-t}
\end{align}
である。
\end{enumerate}

\section{テキスト第4章問2}
\begin{shaded}
$t>0$において
\begin{align}
\boldsymbol{x}'=A(t)\boldsymbol{x}+\boldsymbol{f}(t)
\end{align}
の形の2元連立非同次方程式を考える。
ただし、
\begin{align}
\boldsymbol{f}(t)=\left(
\begin{array}{c}
t\\
1/t
\end{array}
\right)
\end{align}
とする。行列
\begin{align}
V(t)=\left(
\begin{array}{cc}
1/t & t\\
1/t & -t
\end{array}
\right)
\end{align}
が対応する同次方程式の基本行列であるとき、以下の問に答えよ。
ただし、$t,s>0$とする。
\begin{enumerate}
\item 係数行列$A(t)$を求めよ。
\item 対応する同次方程式の解核行列$R(t,s)$を求めよ。
\item 初期条件
\begin{align}
\boldsymbol{x}(1)=\left(
\begin{array}{c}
1\\
1
\end{array}
\right)
\end{align}
を満たす、この非同次方程式の解を求めよ。
\end{enumerate}
\end{shaded}
\begin{enumerate}
\item 係数行列の各成分を
\begin{align}
A(t)=\left(
\begin{array}{cc}
a_{11}(t) & a_{12}(t)\\
a_{21}(t) & a_{22}(t)
\end{array}\right)
\end{align}
として、これらを求める。
基本行列の定義から、
\begin{align}
\boldsymbol{v}_{1}(t)=\left(
\begin{array}{c}
1/t\\
1/t
\end{array}
\right),\quad
\boldsymbol{v}_{2}(t)=\left(
\begin{array}{c}
t\\
-t
\end{array}
\right)
\end{align}
は同次方程式の解であるから、それぞれ代入すれば良い。
$\boldsymbol{v}_{1}(t)$を同次方程式に代入すると、
\begin{align}
\left(
\begin{array}{c}
-1/t^{2}\\
-1/t^{2}
\end{array}\right)=
\left(
\begin{array}{cc}
a_{11}(t) & a_{12}(t)\\
a_{21}(t) & a_{22}(t)
\end{array}\right)
\left(
\begin{array}{c}
1/t\\
1/t
\end{array}\right)
\end{align}
であり、
$\boldsymbol{v}_{2}(t)$を代入すると、
\begin{align}
\left(
\begin{array}{c}
1\\
-1
\end{array}\right)=
\left(
\begin{array}{cc}
a_{11}(t) & a_{12}(t)\\
a_{21}(t) & a_{22}(t)
\end{array}\right)
\left(
\begin{array}{c}
t\\
-t
\end{array}\right)
\end{align}
である。この2つ(式としては4つ)を連立させて解くと、
\begin{align}
A(t)=\left(
\begin{array}{cc}
0 & -1/t\\
-1/t & 0
\end{array}\right)
\end{align}
が得られる。
\item 解核行列$R(t,s)$は基本行列$V(t)$を用いると、
\begin{align}
R(t,s)=V(t)V(s)^{-1}
\end{align}
となる(教科書の定理4.3)。計算すると、
\begin{align}
R(t,s)=\frac{1}{2}\left(
\begin{array}{cc}
s/t+t/s & s/t-t/s\\
s/t-t/s & s/t+t/s
\end{array}\right)
\end{align}
である。
\item 教科書の定理4.5から、初期条件$\boldsymbol{x}(1)$を満たす
解は次で与えられる。
\begin{align}
\boldsymbol{x}(t)=R(t,1)\boldsymbol{x}(1)
+\int_{1}^{t}R(t,r)\boldsymbol{f}(r)dr.
\end{align}
あとはこれを計算すれば良い。答えは、
\begin{align}
\boldsymbol{x}(t)=\frac{1}{3}\left(
\begin{array}{c}
2t^{2}-3t+3+1/t\\
-t^{2}+3t+1/t
\end{array}\right)
\end{align}
である。
\end{enumerate}

\section{テキスト第4章問4}
\begin{shaded}
定数係数微分方程式
\begin{align}
\boldsymbol{x}'=\left(
\begin{array}{cc}
1 & 1\\
-1 & 1
\end{array}\right)\boldsymbol{x}
+\left(
\begin{array}{c}
0\\
\cos\omega t
\end{array}\right)
\end{align}
に対して次の問に答えよ。
\begin{enumerate}
\item 対応する同次方程式の解核行列を求めよ。
\item 初期条件$\boldsymbol{x}(0)=\boldsymbol{x}_{0}\in\mathbb{R}^{2}$
を満たす非同次方程式の解を求めよ。
\end{enumerate}
\end{shaded}
\begin{enumerate}
\item 同次方程式は定数係数の微分方程式になるので、教科書の定理4.9により、
解核行列は次で与えられる。
\begin{align}
R(t,s)=\exp(A(t-s)).
\end{align}
ここで、
\begin{align}
A=\left(
\begin{array}{cc}
1 & 1\\
-1 & 1
\end{array}
\right)
\end{align}
である。行列$A$は行列
\begin{align}
T=\left(
\begin{array}{cc}
1 & -1\\
i & i\\
\end{array}
\right)
\end{align}
を用いて、
\begin{align}
T^{-1}AT=\left(
\begin{array}{cc}
1+i & 0\\
0 & 1-i
\end{array}
\right)
\end{align}
と表される。よって、
\begin{align}
\exp(A(t-s))&=
\frac{1}{2i}\left(
\begin{array}{cc}
1 & -1\\
i & i\\
\end{array}
\right)\left(
\begin{array}{cc}
e^{(1+i)(t-s)} & 0\\
0 & e^{(1-i)(t-s)}
\end{array}
\right)\left(
\begin{array}{cc}
i & 1\\
-i & 1
\end{array}
\right)\\
&=e^{t-s}\left(
\begin{array}{cc}
\cos(t-s) & \sin(t-s)\\
-\sin(t-s) & \cos(t-s)
\end{array}
\right)
\end{align}
となる。

\item 解核行列が得られたので、初期条件$\boldsymbol{x}(0)=\boldsymbol{x}_{0}$
を満たす解は、
\begin{align}
\boldsymbol{x}(t)=R(t,0)\boldsymbol{x}(0)
+\int_{0}^{t}R(t,r)\boldsymbol{f}(r)dr
\end{align}
となる。あとはこれを計算すれば良い。答えは、
\begin{align}
\boldsymbol{x}(t)=e^{t}\left(
\begin{array}{cc}
\cos t & \sin t\\
-\sin t & \cos t
\end{array}\right)
\left(\boldsymbol{x}_{0}
+\frac{1}{\omega^{4}+4}
\left(\begin{array}{c}
\omega^{2}-2\\
\omega^{2}+2
\end{array}\right)
\right)\\
-\frac{1}{\omega^{4}+4}\left(
\begin{array}{c}
(\omega^{2}-2)\cos\omega t+2\omega\sin\omega t\\
(\omega^{2}+2)\cos\omega t-\omega^{3}\sin\omega t
\end{array}\right)
\end{align}
である。
\end{enumerate}
\begin{remark}
対角化するための行列$T$は基底のとり方によって、いくらでも値が取り替わることに注意。
ただ、行列$T$のとり方が違えども、答えは必ず一致するはずなので、
この解答と違う$T$を選んでいたとしても、答えは同じである必要がある。
\end{remark}
\begin{remark}
答えだけを書くと不親切らしいので、計算の過程も書くことにする。
\begin{align}
\boldsymbol{x}(t)&=R(t,0)\boldsymbol{x}(0)
+\int_{0}^{t}R(t,r)\boldsymbol{f}(r)dr\\
&=e^{t}\left(
\begin{array}{cc}
\cos t & \sin t\\
-\sin t & \cos t
\end{array}\right)
\boldsymbol{x}_{0}
+\int_{0}^{t}e^{t-r}\left(
\begin{array}{cc}
\cos(t-r) & \sin(t-r)\\
-\sin(t-r) & \cos(t-r)
\end{array}
\right)
\left(
\begin{array}{c}
0\\
\cos\omega r
\end{array}\right)dr\\
&=e^{t}\left(
\begin{array}{cc}
\cos t & \sin t\\
-\sin t & \cos t
\end{array}\right)
\boldsymbol{x}_{0}
+e^{t}\int_{0}^{t}e^{-r}
\left(
\begin{array}{c}
\sin(t-r)\cos\omega r\\
\cos(t-r)\cos\omega r
\end{array}
\right)dr\\
&=e^{t}\left(
\begin{array}{cc}
\cos t & \sin t\\
-\sin t & \cos t
\end{array}\right)
\boldsymbol{x}_{0}
+e^{t}\left(
\begin{array}{cc}
\sin t & -\cos t\\
\cos t & \sin t
\end{array}
\right)
\left(
\begin{array}{c}
\int_{0}^{t}e^{-r}\cos r\cos\omega rdr\\
\int_{0}^{t}e^{-r}\sin r\cos\omega rdr
\end{array}\right)
\end{align}
となる。よって、次の2つの積分
\begin{align}
I_{1}=\int_{0}^{t}e^{-r}\cos r\cos\omega rdr,\quad
I_{2}=\int_{0}^{t}e^{-r}\sin r\cos\omega rdr
\end{align}
を計算すれば良い。
$I_{1}$を詳しく計算すると、
\begin{align}
I_{1}&=\int_{0}^{t}e^{-r}\cos r\left(\frac{\sin\omega r}{\omega}\right)'dr\\
&=\left[e^{-r}\cos r\frac{\sin\omega r}{\omega}\right]_{0}^{t}
-\frac{1}{\omega}\int_{0}^{t}\left(e^{-r}\cos r\right)'\sin\omega rdr\\
&=\frac{1}{\omega}e^{-t}\cos t\sin\omega t
+\frac{1}{\omega}\int_{0}^{t}e^{-r}\cos r\sin\omega rdr
+\frac{1}{\omega}\int_{0}^{t}e^{-r}\sin r\sin\omega rdr\\
&=\frac{1}{\omega}e^{-t}\cos t\sin\omega t
+\frac{1}{\omega}\int_{0}^{t}e^{-r}\cos r\left(-\frac{\cos\omega r}{\omega}\right)'dr
+\frac{1}{\omega}\int_{0}^{t}e^{-r}\sin r\left(-\frac{\cos\omega r}{\omega}\right)'dr\\
&=\frac{1}{\omega}e^{-t}\cos t\sin\omega t
-\frac{1}{\omega^{2}}\left[e^{-r}\cos r\cos\omega r\right]_{0}^{t}
+\frac{1}{\omega^{2}}\int_{0}^{t}\left(e^{-r}\cos r\right)'\cos\omega rdr\\
&\quad-\frac{1}{\omega^{2}}\left[e^{-r}\sin r\cos\omega r\right]_{0}^{t}
+\frac{1}{\omega^{2}}\int_{0}^{t}\left(e^{-r}\sin r\right)'\cos\omega rdr\\
&=\frac{1}{\omega}e^{-t}\cos t\sin\omega t
-\frac{1}{\omega^{2}}e^{-t}\cos t\cos\omega t
+\frac{1}{\omega^{2}}
-\frac{1}{\omega^{2}}e^{-t}\sin t\cos\omega t
-\frac{2}{\omega^{2}}I_{2}
\end{align}
となる。$I_{2}$も同様に計算すると、
\begin{align}
I_{2}&=\int_{0}^{t}e^{-r}\sin r\left(\frac{\sin\omega r}{\omega}\right)'dr\\
&=\left[e^{-r}\sin r\frac{\sin\omega r}{\omega}\right]_{0}^{t}
-\frac{1}{\omega}\int_{0}^{t}\left(e^{-r}\sin r\right)'\sin\omega rdr\\
&=\frac{1}{\omega}e^{-t}\sin t\sin\omega t
+\frac{1}{\omega}\int e^{-r}\sin r\sin\omega rdr
-\frac{1}{\omega}\int e^{-r}\cos r\sin\omega rdr\\
&=\frac{1}{\omega}e^{-t}\sin t\sin\omega t
+\frac{1}{\omega}\int e^{-r}\sin r\left(-\frac{\cos\omega r}{\omega}\right)'dr
-\frac{1}{\omega}\int e^{-r}\cos r\left(-\frac{\cos\omega r}{\omega}\right)'dr\\
&=\frac{1}{\omega}e^{-t}\sin t\sin\omega t
-\frac{1}{\omega^{2}}\left[e^{-r}\sin r\cos\omega r\right]_{0}^{t}
+\frac{1}{\omega^{2}}\int_{0}^{t}\left(e^{-r}\sin r\right)'\cos\omega rdr\\
&\quad
+\frac{1}{\omega^{2}}\left[e^{-r}\cos r\cos\omega r\right]_{0}^{t}
-\frac{1}{\omega^{2}}\int_{0}^{t}\left(e^{-r}\cos r\right)'\cos\omega rdr\\
&=\frac{1}{\omega}e^{-t}\sin t\sin\omega t
-\frac{1}{\omega^{2}}e^{-t}\sin t\cos\omega t
+\frac{1}{\omega^{2}}e^{-t}\cos t\cos\omega t
-\frac{1}{\omega^{2}}
+\frac{2}{\omega^{2}}I_{1}
\end{align}
となる。この2つをまとめると、
\begin{align}
&\begin{pmatrix}
1 & 2/\omega^{2}\\
-2/\omega^{2} & 1
\end{pmatrix}
\begin{pmatrix}
I_{1}\\
I_{2}
\end{pmatrix}
=
\begin{pmatrix}
-\frac{1}{\omega^{2}}e^{-t}\cos t\cos\omega t
-\frac{1}{\omega^{2}}e^{-t}\sin t\cos\omega t
+\frac{1}{\omega}e^{-t}\cos t\sin\omega t
+\frac{1}{\omega^{2}}\\
\frac{1}{\omega^{2}}e^{-t}\cos t\cos\omega t
-\frac{1}{\omega^{2}}e^{-t}\sin t\cos\omega t
+\frac{1}{\omega}e^{-t}\sin t\sin\omega t
-\frac{1}{\omega^{2}}
\end{pmatrix}
\\
\Leftrightarrow&
\begin{pmatrix}
I_{1}\\ I_{2}
\end{pmatrix}
=\frac{\omega^{4}}{\omega^{4}+4}
\begin{pmatrix}
1 & -2/\omega^{2}\\
2/\omega^{2} & 1
\end{pmatrix}
\begin{pmatrix}
-\frac{1}{\omega^{2}}e^{-t}\cos t\cos\omega t
-\frac{1}{\omega^{2}}e^{-t}\sin t\cos\omega t
+\frac{1}{\omega}e^{-t}\cos t\sin\omega t
+\frac{1}{\omega^{2}}\\
\frac{1}{\omega^{2}}e^{-t}\cos t\cos\omega t
-\frac{1}{\omega^{2}}e^{-t}\sin t\cos\omega t
+\frac{1}{\omega}e^{-t}\sin t\sin\omega t
-\frac{1}{\omega^{2}}
\end{pmatrix}
\end{align}
であるので、あとはこれを代入すれば良い。
\begin{align}
\boldsymbol{x}(t)=&e^{t}
\begin{pmatrix}
\cos t & \sin t\\
-\sin t & \cos t
\end{pmatrix}
\boldsymbol{x}_{0}\\
&+\frac{e^{t}\omega^{4}}{\omega^{4}+4}
\begin{pmatrix}
\sin t & -\cos t\\
\cos t & \sin t
\end{pmatrix}
\begin{pmatrix}
1 & -2/\omega^{2}\\
2/\omega^{2} & 1
\end{pmatrix}
\begin{pmatrix}
-\frac{1}{\omega^{2}}e^{-t}\cos t\cos\omega t
-\frac{1}{\omega^{2}}e^{-t}\sin t\cos\omega t
+\frac{1}{\omega}e^{-t}\cos t\sin\omega t
+\frac{1}{\omega^{2}}\\
\frac{1}{\omega^{2}}e^{-t}\cos t\cos\omega t
-\frac{1}{\omega^{2}}e^{-t}\sin t\cos\omega t
+\frac{1}{\omega}e^{-t}\sin t\sin\omega t
-\frac{1}{\omega^{2}}
\end{pmatrix}\\
=&e^{t}
\begin{pmatrix}
\cos t & \sin t\\
-\sin t & \cos t
\end{pmatrix}
\boldsymbol{x}_{0}\\
+&\frac{e^{t}\omega^{4}}{\omega^{4}+4}
\begin{pmatrix}
\sin t-2\cos t/\omega^{2} & -2\sin t/\omega^{2}-\cos t\\
\cos t+2\sin t/\omega^{2} & -2\cos t/\omega^{2}+\sin t
\end{pmatrix}
\begin{pmatrix}
-\frac{1}{\omega^{2}}e^{-t}\cos t\cos\omega t
-\frac{1}{\omega^{2}}e^{-t}\sin t\cos\omega t
+\frac{1}{\omega}e^{-t}\cos t\sin\omega t
+\frac{1}{\omega^{2}}\\
\frac{1}{\omega^{2}}e^{-t}\cos t\cos\omega t
-\frac{1}{\omega^{2}}e^{-t}\sin t\cos\omega t
+\frac{1}{\omega}e^{-t}\sin t\sin\omega t
-\frac{1}{\omega^{2}}
\end{pmatrix}\\
=&e^{t}
\begin{pmatrix}
\cos t & \sin t\\
-\sin t & \cos t
\end{pmatrix}
\boldsymbol{x}_{0}\\
&+\frac{e^{t}\omega^{4}}{\omega^{4}+4}
\begin{pmatrix}
\sin t & -\cos t\\
\cos t & \sin t
\end{pmatrix}
\begin{pmatrix}
-\frac{1}{\omega^{2}}e^{-t}\cos t\cos\omega t
-\frac{1}{\omega^{2}}e^{-t}\sin t\cos\omega t
+\frac{1}{\omega}e^{-t}\cos t\sin\omega t
+\frac{1}{\omega^{2}}\\
\frac{1}{\omega^{2}}e^{-t}\cos t\cos\omega t
-\frac{1}{\omega^{2}}e^{-t}\sin t\cos\omega t
+\frac{1}{\omega}e^{-t}\sin t\sin\omega t
-\frac{1}{\omega^{2}}
\end{pmatrix}\\
&-\frac{2e^{t}\omega^{2}}{\omega^{4}+4}
\begin{pmatrix}
\cos t & \sin t\\
-\sin t & \cos t
\end{pmatrix}
\begin{pmatrix}
-\frac{1}{\omega^{2}}e^{-t}\cos t\cos\omega t
-\frac{1}{\omega^{2}}e^{-t}\sin t\cos\omega t
+\frac{1}{\omega}e^{-t}\cos t\sin\omega t
+\frac{1}{\omega^{2}}\\
\frac{1}{\omega^{2}}e^{-t}\cos t\cos\omega t
-\frac{1}{\omega^{2}}e^{-t}\sin t\cos\omega t
+\frac{1}{\omega}e^{-t}\sin t\sin\omega t
-\frac{1}{\omega^{2}}
\end{pmatrix}\\
=&e^{t}
\begin{pmatrix}
\cos t & \sin t\\
-\sin t & \cos t
\end{pmatrix}
\boldsymbol{x}_{0}\\
&+\frac{e^{t}\omega^{4}}{\omega^{4}+4}
\begin{pmatrix}
-e^{-t}\cos\omega t/\omega^{2}+(\cos t+\sin t)/\omega^{2}\\
-e^{-t}\cos\omega t/\omega^{2}+e^{-t}\sin\omega t/\omega
+(-\sin t+\cos t)/\omega^{2}
\end{pmatrix}\\
&-\frac{2e^{t}\omega^{2}}{\omega^{4}+4}
\begin{pmatrix}
-e^{-t}\cos\omega t/\omega^{2}+e^{-t}\sin\omega t/\omega
+(-\sin t+\cos t)/\omega^{2}\\
e^{-t}\cos\omega t/\omega^{2}-(\cos t+\sin t)/\omega^{2}
\end{pmatrix}\\
=&e^{t}
\begin{pmatrix}
\cos t & \sin t\\
-\sin t & \cos t
\end{pmatrix}
\boldsymbol{x}_{0}\\
&+\frac{e^{t}}{\omega^{4}+4}
\begin{pmatrix}
(\omega^{2}-2)\cos t+(\omega^{2}+2)\sin t\\
-(\omega^{2}-2)\sin t+(\omega^{2}+2)\cos t
\end{pmatrix}
+\frac{1}{\omega^{4}+4}
\begin{pmatrix}
-\omega^{2}\cos\omega t+2\cos\omega t-2\omega\sin\omega t\\
-\omega^{2}\cos\omega t+\omega^{3}\sin\omega t-2\cos\omega t
\end{pmatrix}\\
=&e^{t}
\begin{pmatrix}
\cos t & \sin t\\
-\sin t & \cos t
\end{pmatrix}\left(
\boldsymbol{x}_{0}
+\frac{1}{\omega^{4}+4}
\begin{pmatrix}
\omega^{2}-2\\
\omega^{2}+2
\end{pmatrix}
\right)
-\frac{1}{\omega^{4}+4}
\begin{pmatrix}
(\omega^{2}-2)\cos\omega t+2\omega\sin\omega t\\
(\omega^{2}+2)\cos\omega t-\omega^{3}\sin\omega t
\end{pmatrix}
\end{align}
となる。
\end{remark}
\end{document}