\documentclass{jsarticle}
\usepackage{amsmath,amssymb}
\usepackage{framed}
\usepackage[hyphens]{url}
\usepackage{ascmac}
\usepackage[dvipdfmx]{graphicx,color}
\usepackage{theorem}
\newtheorem{theorem}{定理}
\newtheorem{proof}{証明}
\newtheorem{eg}{例}
\def\qed{\hfill $\Box$}

\begin{document}
\title{ロジスティック方程式の差分化}
\author{米田亮介}
\maketitle	

\begin{abstract}
ロジスティック方程式の差分化に関する気付きのメモ。
シンプレクティック数値積分と若干似ている??
\end{abstract}
\section{ロジスティック方程式}
微分方程式で記述される可積分系に対して、
\textbf{可積分性を保ったまま差分化したい}、という要請があるらしい。
このことは次の具体例を見るとよく分かる。

\begin{eg}[ロジスティック方程式]
ロジスティック方程式は次で与えられる。
\begin{framed}
\begin{align}
\frac{dy}{dt}=(1-y)y.
\end{align}
\end{framed}
この微分方程式の一般解は定数$C$を用いて、
\begin{align}
y(t)=\frac{1}{1+Ce^{-t}}
\end{align}
と書けることが知られている。
この解は単純な振る舞いをすることが分かる。
次に、この微分方程式を差分化することで数値計算できる形に載せることを試みる。
一番素朴な差分化の手法はオイラー差分である。
オイラー差分は時刻幅$h$として、微分を次のように近似する。
\begin{align}
\frac{dy}{dt}\longrightarrow\frac{y_{n+1}-y_{n}}{h}
\end{align}
このとき差分化されたロジスティック方程式は次のようになる。
\begin{align}
y_{n+1}=y_{n}+h(1-y_{n})y_{n}.
\end{align}
しかし、この離散方程式は本質的にカオスを記述する\textbf{ロジスティックマップ}
になっていて、微分方程式のときと異なって非常に複雑な振る舞いを見せてしまう。
この観点からこの差分化は失敗していると言える。
\qed\end{eg}

このように素朴な差分化では微分方程式の性質を全く保存しないことがある。
そのため、可積分性を保存するような差分化を探すことが一つの目標になっている。
幸い、ロジスティック方程式の場合には微分方程式の解の挙動を保存するような差分化が知られている。
そのために、はじめに\textbf{線形化}
\footnote{ここでいう線形化は、適切な変数変換によって非線形な微分方程式を(厳密に)線形な方程式に
書き換えることである。力学系であるような、定常解の安定性を調べるためにその周りの摂動を考えて
線形化する、ということではない。}
を行う。
次の変数変換を考える。
\begin{align}
y=\frac{1}{1+z}.
\end{align}
すると、ロジスティック方程式は$z$についての線形の方程式に書き換わる。
\begin{align}
\frac{dz}{dt}&=\frac{d}{dt}\left(\frac{1}{y}-1\right)\\
&=-\frac{1}{y^{2}}\frac{dy}{dt}\\
&=\frac{y-1}{y}=-z.
\end{align}
この変数変換で得られる線形の方程式について、オイラー差分を行うと、
\begin{align}
z_{n+1}=(1-h)z_{n}
\end{align}
となる。最後にもとの微分方程式の形に戻すために、
\begin{align}
y_{n}=\frac{1}{1+z_{n}}
\end{align}
という変数変換を行うと、
\begin{framed}
\begin{align}
y_{n+1}=y_{n}+h(1-\textcolor{red}{y_{n+1}})y_{n}
\end{align}
\end{framed}
となる。
この離散方程式はもとの微分方程式と同じような振る舞いを見せてくれる。
注目すべき点はもとの素朴な差分化による離散方程式に比べて、
右辺の一つの$y_{n}$が$y_{n+1}$に変わっただけである、という点である。

\section{シンプレクティック数値積分}
一方で、\textbf{シンプレクティック数値積分}\cite{吉田春夫1990symplectic}がある。
これは、与えられたハミルトニアンに対する系の時間発展を数値計算する際、
変数が常に正準変数となるような差分化の方法である。
ただし、ハミルトニアンは次の形に制限されているとする。
\begin{align}
H(q,p)=T(p)+V(q).
\end{align}
時刻$t_{n}$と$t_{n+1}=t_{n}+\Delta t$における位置と運動量座標を
$(q_{n},p_{n}),(q_{n+1},p_{n+1})$とする。
このとき、ハミルトニアン$H(q,p)$に対する差分化を次のように与える。
\begin{align}
&q_{n+1}=q_{n}+\partial_{p}H(q_{n},p_{n})\Delta t,\\
&p_{n+1}=p_{n}-\partial_{q}H(q_{n},p_{n})\Delta t.
\end{align}
すると、一般には$(q_{n},p_{n}),(q_{n+1},p_{n+1})$は正準変換にはならない。
しかし、この変換を少しだけ変えて、
\begin{framed}
\begin{align}
&q_{n+1}=q_{n}+\partial_{p}H(q_{n},p_{n})\Delta t,\\
&p_{n+1}=p_{n}-\partial_{q}H(\textcolor{red}{q_{n+1}},p_{n})\Delta t.
\end{align}
\end{framed}
とすると、$(q_{n},p_{n}),(q_{n+1},p_{n+1})$は正準変換で結ばれることが示されている。
これは、先のロジスティック方程式において、可積分性を保つような差分化を行うことによって得られた
離散の方程式があるが、そこで$y_{n}\to y_{n+1}$としたものに非常に似ている。
これら2つの間に何らかの関係はないか、非常に気になるところである。
\bibliography{logi}
\bibliographystyle{junsrt}


\end{document}