\documentclass{jsarticle}
\usepackage{amsmath,amssymb}
\usepackage[T1]{fontenc}
\usepackage{newpxtext, newpxmath}
\usepackage{ascmac}
\usepackage{fancybox}
\usepackage[dvipdfmx]{graphicx,color}
\usepackage{theorem}
\newtheorem{theorem}{定理}
\newtheorem{proof}{証明}
\newtheorem{eg}{例}
\def\qed{\hfill $\Box$}
\def\diff{\textrm{d}}

\begin{document}
\title{工業数学A1}
\author{米田亮介}
\maketitle	

\section*{\fbox{$6$}}
$A(z)=(a_{jk}(z))_{1\leq j,k\leq N}$を$N$次正方行列、
$D=\left\{z\in\mathbb{C}\mid |z|<1\right\}$を単位円板、
$m$を正の整数とし、以下の(A),(B)を仮定する。
\begin{enumerate}
\item[(A)] 各$a_{jk}(z)$は$D$上の正則関数である。
\item[(B)] $\det A(z)$は$z=0$に$m$位の零点をもつ。 
\end{enumerate}
このとき、十分に小さい正数$\varepsilon$に対して、次式が成り立つことを示せ。
\begin{align}
m=\mathrm{tr}\left(\frac{1}{2\pi i}\int_{C_{\varepsilon}}A(z)^{-1}\frac{\diff}{\diff z}A(z)~\diff z\right)
\end{align}
ここで積分路$C_{\varepsilon}$は円周$C_{\varepsilon}=\left\{z\in\mathbb{C}\mid |z|=\varepsilon\right\}$
を正の向きに一周するものとし、$\mathrm{tr}(X)$は行列$X$のトレース(trace)を表す。

\section*{こたえ}
まず$N=2$の場合を考えてみる。
\begin{align*}
\begin{aligned}
\mathrm{tr}\left(A(z)^{-1}\frac{\diff}{\diff z}A(z)\right)
&=\mathrm{tr}\left(\frac{1}{a_{11}a_{22}-a_{12}a_{21}}
\begin{pmatrix}
    a_{22} & -a_{12} \\
    -a_{21} & a_{11} \\
\end{pmatrix}
\begin{pmatrix}
    a_{11}' & a_{12}' \\
    a_{21}' & a_{22}' \\
\end{pmatrix}
\right)\\
&=\frac{a_{11}'a_{22}+a_{11}a_{22}'-a_{12}'a_{21}-a_{12}a_{21}'}{a_{11}a_{22}-a_{12}a_{21}}\\
&=\frac{\left(a_{11}a_{22}-a_{12}a_{21}\right)'}{a_{11}a_{22}-a_{12}a_{21}}
=\frac{\left(\det A(z)\right)'}{\det A(z)}
\end{aligned}
\end{align*}
ときれいにまとめることができる。$\det A(z)$は$a_{jk}(z)$の多項式で表されるので正則関数である。
偏角の原理を使えば
\begin{align}
\frac{1}{2\pi i}\int_{C_{\varepsilon}}\mathrm{tr}\left(A(z)^{-1}\frac{\diff}{\diff z}A(z)\right)~\diff z=\frac{1}{2\pi i}\int_{C_{\varepsilon}}\frac{\left(\det A(z)\right)'}{\det A(z)}~\diff z=m
\end{align}
となることがわかる。

一般の$N$についても
\begin{align}
\mathrm{tr}\left(A(z)^{-1}\frac{\diff}{\diff z}A(z)\right)=\frac{\left(\det A(z)\right)'}{\det A(z)}
\end{align}
が成り立つことが予想される。これを示す。

行列$A$の逆行列$A^{-1}=(\tilde{a}_{ij})_{1\leq i,j\leq N}$の各要素は
\begin{align}
\tilde{a}_{ij}=\frac{\Delta_{ji}}{\det A}
\end{align}
で書けることを思い出そう。ここで$\Delta_{ji}$は$A$の$(j,i)$余因子である。
このとき、
\begin{align}
\mathrm{tr}\left(A^{-1}(z)\frac{\diff}{\diff z}A(z)\right)
=\sum_{i,j=1}^{N}\tilde{a}_{ij}a_{ji}'
=\frac{1}{\det A(z)}\sum_{i,j=1}^{N}a_{ji}'\Delta_{ji}
\end{align}
となる。行列式の微分は
\begin{align*}
\begin{aligned}
    \left(\det A(z)\right)'=&\left(\sum_{\sigma\in\mathfrak{S}_{N}}\mathrm{sgn}(\sigma)a_{1\sigma(1)}a_{2\sigma(2)}\cdots a_{N\sigma(N)}\right)'\\
    =&\sum_{\sigma\in\mathfrak{S}_{N}}\mathrm{sgn}(\sigma)a_{1\sigma(1)}'a_{2\sigma(2)}\cdots a_{N\sigma(N)}\\
    &+\sum_{\sigma\in\mathfrak{S}_{N}}\mathrm{sgn}(\sigma)a_{1\sigma(1)}a_{2\sigma(2)}'\cdots a_{N\sigma(N)}\\
    &+\sum_{\sigma\in\mathfrak{S}_{N}}\mathrm{sgn}(\sigma)a_{1\sigma(1)}a_{2\sigma(2)}\cdots a_{N\sigma(N)}'\\
    &=\sum_{i=1}^{N}(A\mathrm{の}i\mathrm{行目のみを微分した行列の行列式})\\
    &=\sum_{i=1}^{N}\left(\sum_{j=1}^{N}a_{ij}'\Delta_{ij}\right)
    =\sum_{i,j=1}^{N}a_{ij}'\Delta_{ij}
\end{aligned}
\end{align*}
と書けるので、
\begin{align}
\mathrm{tr}\left(A^{-1}(z)\frac{\diff}{\diff z}A(z)\right)=\frac{\left(\det A(z)\right)'}{\det A(z)}
\end{align}
となることが示された。以上より、偏角の原理から
\begin{align}
    m=\mathrm{tr}\left(\frac{1}{2\pi i}\int_{C_{\varepsilon}}A(z)^{-1}\frac{\diff}{\diff z}A(z)~\diff z\right)
\end{align}
が示された。

\end{document}