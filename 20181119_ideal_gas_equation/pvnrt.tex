\documentclass{jsarticle}
\usepackage{amsmath,amssymb}
\usepackage{ascmac}
\usepackage[dvipdfmx]{graphicx,color}
\usepackage{theorem}
\newtheorem{theorem}{定理}
\newtheorem{proof}{証明}
\newtheorem{eg}{例}
\newtheorem{remark}{Remark}
\def\qed{\hfill $\Box$}

\begin{document}
\title{理想気体の状態方程式の導出}
\author{米田亮介}
\maketitle	

設定としては$N$個の質点からなる理想気体(質点同士の衝突を考えない気体)
が体積$V$の箱のなかに閉じ込められるときの状態方程式を考える。

この系のハミルトニアンは、相互作用項がないことを考慮して
\begin{align}
H=\sum_{i=1}^{3N}\frac{p_{i}^{2}}{2m}
\end{align}
と書ける。
ここで、各質点は等質量で$m$とした。
このとき、エネルギーが$E$であるときの状態数$\Omega_{0}(E,N,V)$は次のように計算できる。
\begin{align}
\Omega_{0}(E,N,V)=\frac{V^{N}}{h^{3N}N!}\int_{\sum p_{i}^{2}\leq2mE}\prod_{i} dp_{i}s.
\end{align}
この積分は半径$\sqrt{2mE}$の$3N$次元球の体積を表す。
ゆえにこの積分は計算する事ができて、
\begin{align}
\Omega_{0}(E,N,V)=\frac{V^{N}}{h^{3N}N!}\frac{(2\pi mE)^{3N/2}}{\Gamma\left(\frac{3N}{2}+1\right)}
\end{align}
となる。
Stirlingの公式
\begin{align}
N!\simeq\left(\frac{N}{e}\right)^{N}
\end{align}
を用いるとエントロピーは次のように計算できる。
\begin{align}
S(E,N,V)=k\log\Omega_{0}=Nk\left[\log\frac{V}{N}+\frac{3}{2}\log\frac{2E}{3N}
+\log\frac{(2\pi m)^{3/2}e^{5/2}}{h^{3}}\right].
\label{eq:entorpy}
\end{align}
エントロピーに関する次の公式
\begin{align}
&\frac{1}{T}=\left(\frac{\partial S}{\partial E}\right)_{V,N},\\
&\frac{p}{T}=\left(\frac{\partial S}{\partial V}\right)_{E,N}
\end{align}
を思い出す。
式~\eqref{eq:entorpy}をエネルギーで偏微分すると、
\begin{align}
&\frac{1}{T}=Nk\frac{3}{2E}\\
\Longleftrightarrow&kT=\frac{2}{3}\frac{E}{N}=\frac{2}{3}\overline{\varepsilon}
\end{align}
が得られる($\overline{\varepsilon}$はエネルギー密度)。
式~\eqref{eq:entorpy}を体積で偏微分すると、
\begin{align}
&\frac{p}{T}=Nk\frac{1}{V}\\
\Longleftrightarrow&pV=NkT
\end{align}
となり、理想気体の状態方程式が得られた。

\begin{remark}
今回、状態数の計算には久保亮五先生の本の定義に従って$h$で割ることをした。
しかし、状態数としては別に$h$で割る必要はない。
実際、エネルギーに関する式や、状態方程式においては$h$が登場してないことからわかる。	
\end{remark}

\begin{remark}
半径$r$の$N$次元球の体積$C_{N}$はガンマ関数を用いて次のように書ける。
\begin{align}
C_{N}=\frac{\pi^{N/2}}{\Gamma\left(\frac{N}{2}+1\right)}r^{N}.
\end{align}
\end{remark}

\end{document}