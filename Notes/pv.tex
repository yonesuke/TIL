\documentclass{jsarticle}
\usepackage{amsmath,amssymb}
\usepackage{ascmac}
\usepackage[dvipdfmx]{graphicx,color}
\usepackage{theorem}
\newtheorem{theorem}{定理}
\newtheorem{proof}{証明}
\newtheorem{eg}{例}
\def\qed{\hfill $\Box$}

\begin{document}
\title{Cauchyの主値積分}
\author{米田亮介}
\maketitle	

Cauchyの主値積分は次で与えられる。
有限区間の場合、
\begin{align}
PV\int_{a}^{b}f(x)dx=\lim_{\varepsilon\to 0}\left(
\int_{a}^{c-\varepsilon}f(x)dx+\int_{c+\varepsilon}^{b}f(x)dx
\right)
\end{align}
である。また、無限区間の場合、
\begin{align}
PV\int_{-\infty}^{\infty}f(x)dx=\lim_{R\to\infty}\left(
\int_{-R}^{R}f(x)dx
\right)
\end{align}
である。

\begin{eg}
\begin{align}
PV\int_{-1}^{1}\frac{dx}{x}=\lim_{\varepsilon\to0}\left(
\int_{-1}^{-\varepsilon}\frac{dx}{x}+\int_{\varepsilon}^{1}\frac{dx}{x}
\right)=0
\end{align}
\end{eg}

\begin{eg}
\begin{align}
PV\int_{-1}^{1}\frac{dx}{\sqrt[3]{x^2}}&=\lim_{\varepsilon\to0}\left(
\int_{-1}^{-\varepsilon}\frac{dx}{\sqrt[3]{x^2}}+\int_{\varepsilon}^{1}\frac{dx}{\sqrt[3]{x^2}}
\right)\\
&=\lim_{\varepsilon\to0}\left(
3\left[\sqrt[3]{x}\right]_{-1}^{-\varepsilon}
+3\left[\sqrt[3]{x}\right]_{\varepsilon}^{1}\right)\\
&=6
\end{align}
\end{eg}


\end{document}