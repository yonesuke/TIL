\documentclass[fleqn]{jsarticle}
\usepackage[dvipdfmx]{graphicx,color}
\usepackage{ulem}
\usepackage{ascmac}
\usepackage{pxjahyper}
\usepackage{pgf,tikz}
\usepackage{here}
%\usepackage{subcaption}
\usepackage{amsmath,amssymb}
\usepackage{ascmac}
\usepackage{comment}
\usepackage{theorem}
\newtheorem{theorem}{定理}
\newtheorem{proof}{証明}
\newtheorem{remark}{Remark}
\usepackage{framed}
\def\qed{\hfill $\Box$}
\definecolor{shadecolor}{gray}{0.80}

\usepackage{showkeys}

\begin{document}
\title{微分積分学俗論II}
\maketitle
\section{11/14の微積続論II小テスト}
\begin{shaded}
次の微分方程式を解け。
\begin{enumerate}
\item 
\begin{eqnarray}
\frac{dx}{dt}=x\log t
\end{eqnarray}
\item 
\begin{eqnarray}
\frac{dx}{dt}=x\log t+t^{t}
\end{eqnarray}
\end{enumerate}
\end{shaded}
\begin{enumerate}
\item 
$x=0$は解である。
$x\ne 0$のとき両辺を変数分離すると、
\begin{eqnarray}
\frac{dx}{x}=\log tdt
\end{eqnarray}
であり、これを積分すると、
\begin{eqnarray}
\log |x|=t\log t-t+C'
\end{eqnarray}
である。ここで$C'$は積分定数である。
これより、解は
\begin{eqnarray}
|x(t)|=Ce^{t\log t-t}=Ce^{-t}t^{t}
\end{eqnarray}
となる。ここで$C>0$は定数。
$x=0$の解もまとめると、任意の$C\in\mathbb{R}$に対して
\begin{eqnarray}
x(t)=Ce^{-t}t^{t}
\end{eqnarray}
となる。
\item
この問題には定数変化法を使う。
つまり、解の形を
\begin{eqnarray}
x(t)=C(t)e^{-t}t^{t}
\end{eqnarray}
の形に決め打ちして、$C(t)$を求めるという算段である。
このとき、
\begin{eqnarray}
\dot{C}(t)e^{-t}t^{t}+x\log t=x\log t+t^{t}
\end{eqnarray}
である。これより、$C(t)$に関しての微分方程式
\begin{eqnarray}
\dot{C}(t)=e^{t}
\end{eqnarray}
が得られる。故に
\begin{eqnarray}
C(t)=e^{t}+C
\end{eqnarray}
となる。ここで$C$は積分定数である。
以上より、はじめの微分方程式の解は
\begin{eqnarray}
x(t)=(e^{t}+C)e^{-t}t^{t}=t^{t}+Ce^{-t}t^{t}
\end{eqnarray}
と求まった。
\end{enumerate}

\begin{remark}
微分方程式を解くときに思考停止で変数分離法とか定数変化法を使ってるけど、
解が本当にこれだけなのか?ということは常にしっかりと考えるべきことなのかもしれない。
初期値を与えたときに、解の存在と一意性を提示してくれるものは
ベクトル場のリプシッツ連続性であった。
なので、ひとたびベクトル場のリプシッツ連続性を確認することができたならば、
変数分離法や定数変化法は強力な手法になるのである。
今回の問題では自励系の微分方程式ではないので、
補助変数を用意して強引に自励系の多変数微分方程式系にする必要がある。
このとき、高次元にベクトル場が与えられたときにも解の存在と一意性を決めるのは
リプシッツ連続性になるのだろうか?
このことについても少し調べてみる必要がありそうだ。
\end{remark}

\section{テキスト第3章問5}
\begin{shaded}
$a_{j}(t)(j=1,2)$と$q(t)$を$t$のある関数として2階非同次方程式
\begin{eqnarray}
x''+a_{1}(t)x'+a_{2}(t)x=q(t)
\end{eqnarray}
を考える。$x=t,t^{2},t^{3}$が解であるとき、次の問いに答えよ。
\begin{enumerate}
\item 関数$a_{j}(t)$と$q(t)$を定めよ。
\item 一般解を求めよ。
\end{enumerate}
\end{shaded}
\begin{enumerate}
\item $x=t,t^{2},t^{3}$を代入して連立方程式を解くだけ。
\begin{eqnarray}
a_{1}(t)=-\frac{2(2t-1)}{t(t-1)},\\
a_{2}(t)=\frac{2(3t^{2}-3t+1)}{t^{2}(t-1)^{2}},\\
q(t)=\frac{2t}{(t-1)^{2}}
\end{eqnarray}
\item 非同時方程式の一般解は、同次方程式の一般解と特殊解の和になる。
そのために同次方程式
\begin{eqnarray}
x''+a_{1}(t)x'+a_{2}(t)x=0
\end{eqnarray}
を解きたいが、そのまま解くのは難しい。
そこで次のように考える。
非同次方程式の解$t,t^{2}$を同次方程式の基本解$x_{1},x_{2}$
を用いて表してみる。
特殊解は$t^{3}$であるから、
\begin{eqnarray}
&t=c_{1}x_{1}+c_{2}x_{2}+t^{3},\\
&t^{2}=c_{1}'x_{1}+c_{2}'x_{2}+t^{3}
\end{eqnarray}
とかける。
ここで$c_{1},c_{2},c_{1}',c_{2}'$は適当な定数である。
両辺を引くと、
\begin{eqnarray}
t-t^{2}=(c_{1}-c_{1}')x_{1}+(c_{2}-c_{2}')x_{2}
\end{eqnarray}
となる。
これより、$t-t^{2}$は基本解の線形和で表せることがわかった。
よって、$t-t^{2}$も同次方程式の解である。
同様の考えによって、$t-t^{3}$も同次方程式の解である。

$t-t^{2},t-t^{3}$はそれぞれ独立であるから、
同次方程式の基本解になる。
非同次方程式の特殊解として$t^{3}$を選ぶと、一般解は
\begin{eqnarray}
c_{1}(t-t^{2})+c_{2}(t-t^{3})+t^{3}
\end{eqnarray}
になる。
\end{enumerate}
\begin{remark}
最後の答えを書くところで非同次方程式の特殊解として、$t^{3}$を選んだが、
$t,t^{2}$でも良い。
どれを選んでも結局$c_{1},c_{2}$の任意性から解空間はおなじになる。
\end{remark}

\begin{remark}
よくある問題は特殊解を一つ与えた上で、
非同次方程式の一般解を求めさせる問題だが、
この問題は特殊解を(複数個)
先に与えている、という点において目新しい。
よくよく考えると、微分方程式を与えると、
(一般解に対応する)解空間を考えることができるが、
この問題は、解空間の中の有限個の点から解空間全体を復元できることも主張している。
線形空間であれば、次元の数のぶんだけ解を用意すればよいが、
今回は線形空間でもない、という点においても非常に面白い問題であると思う。
\end{remark}
\end{document}