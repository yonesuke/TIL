\documentclass{jsarticle}
\usepackage[dvipdfmx]{graphicx,color}
\usepackage{ulem}
\usepackage{ascmac}
\usepackage{pgf,tikz}
\usepackage{here}
%\usepackage{subcaption}
\usepackage{amsmath,amssymb}
\usepackage{ascmac}
\usepackage{comment}
\usepackage{theorem}
\newtheorem{theorem}{定理}
\newtheorem{proof}{証明}
\newtheorem{remark}{Remark}
\usepackage{framed}
\def\qed{\hfill $\Box$}
\definecolor{shadecolor}{gray}{0.80}

%\usepackage{showkeys}

\begin{document}
\title{数理解析特論レポート}
\author{米田亮介}
\date{}
\maketitle
\section{Darboux変換}
\begin{shaded}
\begin{enumerate}
\item 差分演算子$L,B$をそれぞれ
$L=e^{\partial/\partial n}+u_{n}e^{-\partial/\partial n},
B=-u_{n}u_{n-1}e^{-2\partial/\partial n}$で定める。
このとき、関数$\varphi_{n}(t)$に対する線形方程式
\begin{align}
L\varphi_{n}(t)=(\lambda+1/\lambda)\varphi_{n}(t),\quad
\frac{d}{dt}\varphi_{n}(t)=B\varphi_{n}(t)
\end{align}
の両立条件から、無限格子上の連続時間発展方程式であるLotka--Voltera(LV)方程式
\begin{align}
\frac{d}{dt}u_{n}=u_{n}(u_{n+1}-u_{n-1})
\end{align}
が導かれることを示せ。
ここで、$\lambda+1/\lambda$は$L$の固有値を与える定数。
\item LV方程式の自明な解$u=1$をseed(種)として、
Darboux変換を$k$回適用することで得られる差分作用素を
$L^{(k)}=e^{\partial/\partial n}+u_{n}^{(k)}e^{-\partial/\partial n}$
で表す。
このとき、$u_{n}^{(1)}$および$u_{n}^{(2)}$を求めよ。
\end{enumerate}
\end{shaded}
関数$f$のシフト演算子$e^{a\partial/\partial x}$は
\begin{align}
e^{a\partial/\partial x}f(x)=f(x+a)
\end{align}
で書けるのであった\footnote{
	直観的にはテイラー展開によって説明できる。
	$C^{\omega}$級の関数$f$について
	\begin{align*}
		e^{a\partial/\partial x}f(x)
		=\sum_{n=0}^{\infty}\frac{a^{n}}{n!}\left(\frac{\partial}{\partial x}\right)^{n}f(x)
		=\sum_{n=0}^{\infty}\frac{a^{n}}{n!}f^{(n)}(x)
		=f(x+a)
	\end{align*}
	と形式的に書ける。
}。
本レポートでは関数のシフト演算子の類似として数列$u_{n}$に作用するシフト演算子$e^{k\partial/\partial n}$は
\begin{align}
e^{k\partial/\partial n}u_{n}=u_{n+k}
\end{align}
で作用するように決めることにする。
\begin{enumerate}
\item 線形方程式
\begin{align}
L\varphi_{n}(t)=(\lambda+1/\lambda)\varphi_{n}(t)
\end{align}
の両辺を$t$微分すると、
\begin{align}
L_{t}\varphi_{n}(t)
&=(\lambda+1/\lambda)\frac{d}{dt}\varphi_{n}(t)
-L\frac{d}{dt}\varphi_{n}(t)\\
&=(\lambda+1/\lambda)B\varphi_{n}(t)-LB\varphi_{n}(t)\\
&=B\left[(\lambda+1/\lambda)\varphi_{n}(t)\right]-LB\varphi_{n}(t)\\
&=(BL-LB)\varphi_{n}(t)
\end{align}
が成り立つ必要がある。
\end{enumerate}
\section{離散系・超離散系}
\begin{shaded}
離散Lotka--Voltera(dLV)方程式
\begin{align}
\frac{V_{n}^{(t+1)}}{V_{n}^{(t)}}=\frac{1+\delta V_{n-1}^{(t)}}{1+\delta V_{n+1}^{t+1}}
\end{align}
について考える。
\begin{enumerate}
\item 双線形dLV方程式
\begin{align}
(1+\delta)\tau_{n-1}^{(t+1)}\tau_{n}^{(t)}=\tau_{n-1}^{(t)}\tau_{n}^{(t+1)}+\delta\tau_{n-2}^{(t)}\tau_{n+1}^{(t+1)}
\end{align}
に対して、従属変数変換
\begin{align}
V_{n}^{(t)}=\frac{\tau_{n-1}^{(t)}\tau_{n+2}^{(t+1)}}{\tau_{n}^{(t)}\tau_{n+1}^{(t+1)}}
\end{align}
を施すことで、dLV方程式を導出せよ。
\item 双線形dLV方程式が1ソリトン解
\begin{align}
&\tau_{n}^{(t)}=1+\alpha p^{n}q^{t},\\
&q=\frac{1+\delta(1+\delta)^{-1}p^{-1}}{1+\delta(1+\delta)^{-1}p}
\end{align}
を満たすことを確かめよ。
\item dLV方程式から超離散Lotka--Voltera方程式を導出せよ。
\item 双線形dLV方程式の1ソリトン解$\tau_{n}^{(t)}$の超離散化から
超離散Lotka--Voltera方程式の1ソリトン解$T_{n}^{(t)}$を導出せよ。
\item dLV方程式の1ソリトン解に対して次のパラメータ変換を導入する。
\begin{align}
p=\exp(P/\varepsilon),\quad
\delta=\exp(-1/\varepsilon),\quad
\alpha=\exp(A/\varepsilon).
\end{align}
このとき、$\varepsilon=0.1,0.05,0.001$などと$\varepsilon$を選び、
離散Lotka--Voltera方程式の解を用いて$U_{n}^{(t)}=\varepsilon\log u_{n}^{(t)}$を
プロットすることで、超離散系への移行の様子を観察せよ。
\end{enumerate}
\end{shaded}
\end{document}