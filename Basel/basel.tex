\documentclass{jsarticle}
\usepackage{amsmath,amssymb}
\usepackage[dvipdfmx]{graphicx,color}
\usepackage{url}

\title{$\displaystyle\sum_{n=1}^{\infty}\frac{1}{n^{2}}=\frac{\pi^{2}}{6}$}

\begin{document}
\maketitle

\abstract{バーゼル問題の証明をできる限り記していく。}

\tableofcontents

\newpage

\section{Fourier級数展開を用いる方法}
区間$[-\pi,\pi]$で
\begin{equation}
	f(t)=t^{2}
\end{equation}
となる周期関数$f$を考える。$f$を複素フーリ級数展開すると、
\begin{equation}
	t^{2}=\frac{\pi^{2}}{3}+\sum_{n\ne 0,n=-\infty}^{\infty}\frac{2(-1)^{n}}{n^{2}}e^{-int}
	\label{eq:t^2-fourier}
\end{equation}
と計算できる。ここに$t=\pi$を代入すると、
\begin{align}
	\pi^{2}&=\frac{\pi^{2}}{3}+\sum_{n\ne 0,n=-\infty}^{\infty}\frac{2(-1)^{n}}{n^{2}}e^{-in\pi}\\
	&=\frac{\pi^{2}}{3}+\sum_{n\ne 0,n=-\infty}^{\infty}\frac{2(-1)^{n}}{n^{2}}(-1)^{n}\\
	&=\frac{\pi^{2}}{3}+\sum_{n\ne 0,n=-\infty}^{\infty}\frac{2}{n^{2}}\\
	&=\frac{\pi^{2}}{3}+4\sum_{n=1}^{\infty}\frac{1}{n^{2}}
\end{align}
となる。これより、
\begin{equation}
	\sum_{n=1}^{\infty}\frac{1}{n^{2}}=\frac{\pi^{2}}{6}
\end{equation}
であることが示された\footnote{\url{http://otaku-of-suri.hatenablog.com/entry/2016/06/17/130215}}。

\newpage

\section{Fourier級数展開を用いる方法}
式~\eqref{eq:t^2-fourier}において$t=0$を代入すると、
\begin{equation}
	0=\frac{\pi^{2}}{3}+4\sum_{n=1}^{\infty}\frac{(-1)^{n}}{n^{2}}
\end{equation}
が得られる。この式を整理すると、
\begin{equation}
	\sum_{n=1}^{\infty}\frac{(-1)^{n+1}}{n^{2}}=\frac{\pi^{2}}{12}
\end{equation}
となる。

一方で、$\sum\frac{1}{n^2}$を求めたいが、これは$n\geq 2$において、
\begin{equation}
	\frac{1}{n^{2}}<\frac{1}{n(n-1)}=\frac{1}{n-1}-\frac{1}{n}
\end{equation}
となることを用いると、$\sum\frac{1}{n^2}<2$が示される。
よって、この級数は正項級数で上に有界であるから収束する。
この収束値を$\alpha$とおく。このとき、

\begin{align}
	\alpha-\frac{\pi^{2}}{12}
	&=\sum_{n=1}^{\infty}\frac{1}{n^2}-\sum_{n=1}^{\infty}\frac{(-1)^{n+1}}{n^{2}}\\
	&=\sum_{n=1}^{\infty}\left\{\frac{1}{n^{2}}-\frac{(-1)^{n+1}}{n^{2}}\right\}\\
	&=\sum_{n:偶数}\frac{2}{n^{2}}\\
	&=\sum_{m=1}^{\infty}\frac{2}{(2m)^{2}}\\
	&=\frac{1}{2}\sum_{n=1}^{\infty}\frac{1}{n^{2}}=\frac{\alpha}{2}
\end{align}
これより、
\begin{equation}
	\alpha=\sum_{n=1}^{\infty}\frac{1}{n^{2}}=\frac{\pi^{2}}{6}
\end{equation}
が示された\footnote{\url{http://otaku-of-suri.hatenablog.com/entry/2016/06/17/130215}}。

\end{document}