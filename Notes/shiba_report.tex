\documentclass{jsarticle}
\usepackage[dvipdfmx]{graphicx,color}
\usepackage{ulem}
\usepackage{ascmac}
\usepackage{pgf,tikz}
\usepackage{here}
%\usepackage{subcaption}
\usepackage{amsmath,amssymb}
\usepackage{ascmac}
\usepackage{comment}
\usepackage{theorem}
\newtheorem{theorem}{定理}
\newtheorem{proof}{証明}
\newtheorem{remark}{Remark}
\usepackage{framed}
\def\qed{\hfill $\Box$}
\definecolor{shadecolor}{gray}{0.80}

%\usepackage{showkeys}

\begin{document}
\title{数理物理学通論レポート}
\author{米田亮介}
\date{}
\maketitle
\section*{問題1}
\begin{shaded}
電磁場における電荷の運動は
\begin{align}
m\ddot{\pmb{q}}=e\pmb{E}(\pmb{q},t)+\frac{e}{c}\dot{\pmb{q}}\times\pmb{B}(\pmb{q},t),\qquad(\pmb{q}\in \mathbb{R}^{3})
\end{align}
で表される。
ここで、$e$は電荷、$c$は光速、$\pmb{E},\pmb{B}:\mathbb{R}^{3}\times\mathbb{R}\to\mathbb{R}^{3}$
はそれぞれ電場、磁場である。
\begin{align}
\mathbb{B}=\nabla\times\pmb{A},\quad
\pmb{E}=-\nabla\phi-\frac{1}{c}\frac{\partial\pmb{A}}{\partial t}
\end{align}
を満たす$\pmb{A}$と$\phi$(このペアを電磁ポテンシャルという)が存在すると仮定すると、ハミルトニアン
\begin{align}
H(\pmb{q},\pmb{p},t)=\frac{1}{2m}\left|\pmb{p}-\frac{e}{c}\pmb{A}(\pmb{q},t)\right|^{2}+e\phi(\pmb{q},t)
\end{align}
の正準方程式が上記の微分方程式と一致することを確かめよ。
\end{shaded}
$\pmb{q},\pmb{p}$の各成分を$q_{i},p_{i}(i=1,2,3)$
とベクトルの各成分を書き下すと、ハミルトニアンは
\begin{align}
H=\frac{1}{2m}\left(p_{j}p_{j}-\frac{2e}{c}p_{j}A_{j}+\frac{e^{2}}{c^{2}}A_{j}A_{j}\right)+e\phi
\end{align}
となる。ここで、同じ添字については和を取ることにする
(上付き、下付きとかそういうことはここでは考えない。)。
正準方程式は、
\begin{align}
\begin{split}
\dot{q_{i}}&=\frac{\partial H}{\partial p_{i}}
=\frac{1}{2m}\left(2p_{j}\delta_{i,j}-\frac{2e}{c}A_{j}\delta_{i,j}\right)
=\frac{1}{m}\left(p_{i}-\frac{e}{c}A_{i}\right),
\end{split}\\
\begin{split}
\dot{p_{i}}&=-\frac{1}{2m}\left(-\frac{2e}{c}p_{j}\partial_{i}A_{j}
+\frac{2e^{2}}{c^{2}}A_{j}\partial_{i}A_{j}\right)
-e\partial_{i}\phi\\
&=\frac{e}{mc}\left(p_{j}
-\frac{e}{c}A_{j}\right)\partial_{i}A_{j}-e\partial_{i}\phi
\end{split}
\end{align}
となる。まとめると、
\begin{align}
\begin{split}
m\ddot{q_{i}}&=\dot{p_{i}}-\frac{e}{c}\left(\partial_{t}A_{i}+\dot{q_{j}}\partial_{j}A_{i}\right)\\
&=\frac{e}{mc}\left(p_{j}
-\frac{e}{c}A_{j}\right)\partial_{i}A_{j}
-\frac{e}{c}\dot{q_{j}}\partial_{j}A_{i}
+e\left(-\partial_{i}\phi-\frac{1}{c}\partial_{t}A_{i}\right)\\
&=\frac{e}{c}\left(\dot{q_{j}}\partial_{i}A_{j}-\dot{q_{j}}\partial_{j}A_{i}\right)+eE_{i}
\end{split}
\end{align}
である。これより、
\begin{align}
m\ddot{\pmb{q}}=e\pmb{E}(\pmb{q},t)+\frac{e}{c}\dot{\pmb{q}}\times\pmb{B}(\pmb{q},t)
\end{align}
が示された\footnote{
外積をエディントンのイプシロンを用いて展開すると、
\begin{align*}
(\dot{\pmb{q}}\times\pmb{B})_{i}&=\epsilon_{ijk}\dot{q_{j}}(\pmb{B})_{k}
=\epsilon_{ijk}\dot{q_{j}}(\nabla\times\pmb{A})_{k}
=\epsilon_{ijk}\dot{q_{j}}\epsilon_{klm}\partial_{l}A_{m}
=\epsilon_{ijk}\epsilon_{lmk}\dot{q_{j}}\partial_{l}A_{m}\\
&=\left(\delta_{il}\delta_{jm}-\delta_{im}\delta_{jl}\right)\dot{q_{j}}\partial_{l}A_{m}
=\dot{q_{j}}\partial_{i}A_{j}-\dot{q_{j}}\partial_{j}A_{i}
\end{align*}
となることを用いた。
}。


\section*{問題10}
\begin{shaded}
数列$a_{n}$を$2^{n}$を10進法で表したときの最初の数とする。
つまり、$a_{1}=2,a_{2}=4,a_{3}=8,a_{4}=1,a_{5}=3,\cdots$である。
この数列に$7$が現れることを示せ。
より一般に、任意の$N$と任意の$N$桁の自然数を与えたとき、
$2^{n}$の上位$N$桁がその自然数と一致するような$n$が存在することを示せ。
\end{shaded}



\section*{問題11}
\begin{shaded}
フィボナッチ数列について前問と同様のことが成り立つことを示せ。
\end{shaded}
\section*{問題13}
\begin{shaded}
ケプラー問題を一般化した
\begin{align}
V(r)=ar^{k}
\end{align}
のポテンシャル系の作用-角変数を求めよ。
ここで$k$は整数、$a$は実数で、$ak>0$であるとする。
\end{shaded}
\end{document}