\documentclass{article}
\usepackage{amsmath,amssymb}
\usepackage[T1]{fontenc}
\usepackage{newpxtext, newpxmath}
\usepackage{ascmac}
\usepackage{fancybox}
\usepackage[dvipdfmx]{graphicx,color}

\def\d{\textrm{d}}

\begin{document}
\title{Critical Exponent on the Generalized Kuramoto Model}
\author{Ryosuke Yoneda}
\maketitle	

We consider the \textbf{generalized Kuramoto model}
\begin{align}
&\frac{\d\theta_{j}}{\d t}=\omega_{j}+\frac{K}{N}\sum_{k=1}^{N}\Gamma(\theta_{k}-\theta_{j})-a\sin\theta_{j},\\
&\Gamma(\theta)=\sin\theta+h\sin2\theta,
\end{align}
where $\omega_{j}$ is the natural frequency,
each of which is obeyed by a distribution $g(\omega)$, say \textbf{Lorentz distribution}
\begin{align}
g(\omega)=\frac{\gamma}{\pi}\frac{1}{\omega^{2}+\gamma^{2}},
\end{align}
and $K$ is the coupling strength.
$\Gamma(\theta)$ is the coupling function and $a$ represents the strength
of the external force.

Our main motivation is to observe the $K$-dependency of $r$,
where $r$ is the magnitude of the order parameter $z$:
\begin{align}
z_{n}=re^{i\varphi}=\frac{1}{N}\sum_{k=1}^{N}e^{ni\theta_{k}}.
\end{align}
For $n=1$, $z_{1}$ describes the center of the oscillators:
If $r\sim 0$, the center is close to the origin and
the oscillators is expected to be distributed uniformly
on the circle.
If $r$ is close to $1$, the center is nearly on the circle,
hence the oscillators are close to each other.
Therefore, computing the $K$-dependency of $r$
allows us to capture the relationship between
the coupling strength of the oscillators
and the extent of the synchronization.

The number of the oscillators are conserved throughout
the time evolution in the Kuramoto model.
Therefore, in the limit of the population $N\to\infty$,
we obtain the equation of continuity,
\begin{align}
&\frac{\partial F}{\partial t}+\frac{\partial}{\partial\theta}
\left(v[F]F\right)=0,\\
&v[F]=\omega+K\int_{\mathbb{S}^{1}}\int_{\mathbb{R}}
\Gamma(\theta'-\theta)F(\theta',\omega',t)\d\theta'\d\omega'-a\sin\theta,
\end{align}
where $F(\theta,\omega,t)$ denotes the probability density function
of $(\theta,\omega)$ at time $t$.
Since $\omega$ is obeyed by the distribution function $g(\omega)$,
we have
\begin{align}
g(\omega)=\int_{\mathbb{S}^{1}}F(\theta,\omega,t)\d\theta\d\omega.
\end{align}
The order parameter $z,w$ in the population limit reads
\begin{align}
&z_{n}=\int_{\mathbb{S}^{1}}\int_{\mathbb{R}}e^{ni\theta}F(\theta,\omega,t)\d\theta\d\omega.
\end{align}
Rewriting the vector field $v$ using the order parameter,
we have
\begin{align}
v[F]=\omega+\frac{K}{2i}\left[z_{1}e^{-i\theta}-z_{1}^{*}e^{i\theta}
+h\left(z_{2}e^{-2i\theta}-z_{2}^{*}e^{2i\theta}\right)\right]
-\frac{a}{2i}\left(e^{i\theta}-e^{-i\theta}\right).
\end{align}

When $a=0$, the Kuramoto model has the stationary state solution
\begin{align}
f^{0}(\omega)=\frac{g(\omega)}{2\pi}.
\end{align}
In this state, $r$ becomes zero,
hence the oscillators are uniformly distributed on the circle.
We take care of the perturbation from the stationary state,
so we set 
\begin{align}
	F=f^{0}+f
\end{align}
and substitute it to the equation of continuity.
Taking the linear part of $f$, we have
\begin{align}
&\frac{\partial f}{\partial t}
+\omega\frac{\partial f}{\partial\theta}
+f^{0}\frac{\partial}{\partial\theta}v_{1}[f]
%+\frac{\partial}{\partial\theta}(v_{1}[f]f)
=0,\\
&v_{1}[f]=\frac{K}{2i}\left[ze^{-i\theta}-z^{*}e^{i\theta}
+h\left(we^{-2i\theta}-w^{*}e^{2i\theta}\right)\right]
-\frac{a}{2i}\left(e^{i\theta}-e^{-i\theta}\right)
\end{align}

Let us introduce the Fourier series expansion
\begin{align}
f(\theta,\omega,t)=\sum_{n=-\infty}^{\infty}\tilde{f_{n}}(\omega,t)e^{ni\theta}.
\end{align}

\end{document}